\section{Meilensteine}

In der nachfolgenden Tabelle \ref{tab:Meilenstein Liste} sind alle einzelnen Meilensteine aufgelistet.
Jeder Meilenstein wird noch etwas genauer weiter unten erklärt.

\begin{table}[H]
    \begin{center}
        \begin{tabular}{|p{180px}|c|}
            \hline
            \begin{tabular}{l}
                \\
                \large Recherche \\
                \\
            \end{tabular}
             &
            \begin{tabular}{l}
                \\
                \large \bfseries 13.10.21 \\
                \\
            \end{tabular}
            \\
            \hline
            \begin{tabular}{l}
                \\
                \large Zwischenpräsentation \\
                \\
            \end{tabular}
             &
            \begin{tabular}{l}
                \\
                \large \bfseries 22.10.21 \\
                \\
            \end{tabular}
            \\
            \hline
            \begin{tabular}{l}
                \\
                \large Implementation \\
                \\
            \end{tabular}
             &
            \begin{tabular}{l}
                \\
                \large \bfseries 25.10.21 \\
                \\
            \end{tabular}
            \\
            \hline
            \begin{tabular}{l}
                \\
                \large MVP \\
                \\
            \end{tabular}
             &
            \begin{tabular}{l}
                \\
                \large \bfseries  02.12.21 \\
                \\
            \end{tabular}
            \\
            \hline
            \begin{tabular}{l}
                \\
                \large Endpräsentation und \\
                \large Enddokumentation\\
                \\
            \end{tabular}
             &
            \begin{tabular}{l}
                \large \bfseries 14.01.22 \\
            \end{tabular}
            \\
            \hline
        \end{tabular}
        \caption{Meilenstein Liste}
        \label{tab:Meilenstein Liste}
    \end{center}
\end{table}



\subsection{Recherche 13.10.21}
Während der Recherche haben wir alle Informationen gesammelt, die wir für das Projekt benötigen.
Die Themen Node.js, Keycloak, Angular, Socket.io und NLP waren vorrangige Themen, um Risiken zu minimieren.
Deswegen haben wir uns in dieser Phase möglichst ausführlich informiert und Bücher, Webadressen
und weitere Materialien besorgt. Außerdem haben wir nach Möglichkeit alle Unklarheiten geklärt.

\subsection{Zwischenpräsentation 22.10.21}
Wir haben eine Woche früher (12.10.21) angefangen alle relevanten Themen zu sammeln, um Materialien für die Präsentation zu haben.
Damit die Präsentation sehr interessant für alle Teilnehmer ist, haben wir die wichtigsten Themen optisch ansehbar gestaltet.
In unserer Zeitplanung ist auch die praktische Übung der Folien im Team eingeplant.

\subsection{Implementation 25.10.21}
In der Implementierung wollen wir die recherchierten Materialien umsetzen und praktische Erfahrung sammeln.
Während wir versuchen alle relevanten Informationen in einen MVP umzusetzen. Zusätzlich wird in dieser Phase
ein Teil der Recherche in das \LaTeX-Format übertragen.

\subsection{MVP 02.12.21}
Der MVP ist unser angestrebtes Ziel. Damit wir ein Produkt zum Präsentieren haben.
Während wir unser angesammeltes Wissen in die Praxis umsetzen versuchen wir
frühzeitig ein funktionierendes Produkt mit den Mindestanforderungen umzusetzen.
Wir hatten vorerst geplant unseren MVP zum 03.12.21 zu liefern.
Interessanterweise wurde später der Termin des MVP vom Professor auf den 02.12.21 gelegt.
Wodurch wir unseren geplanten Zeitraum weiterhin nachverfolgen können und
den zeitlichen Rahmen minimal korrigieren müssen. 
Demnach haben wir relativ gut eingeschätzt bis wann der MVP fertig sein sollte.

\subsection{Endpräsentation und Enddokumentation 14.01.22}
Für die Endpräsentation ist eine Woche früher (07.01.22) der Beginn der Erstellung der Präsentation eingeplant.
Ziel ist hierbei, dass ein Prototyp mit interessanten Features und Funktionen vorgestellt werden kann.
Für die Enddokumentation werden wir ab dem Start der Implementation anfangen, alle wichtigen Informationen zu dokumentieren.
Sehr wichtig ist hierbei für unser Team, dass wir alle kontinuierlich wichtige Themen zu unserem Projekt dokumentieren.
Für unsere Dokumentation wählen wir das \LaTeX-Format, da es uns hilft besser kooperativ über Gitlab zu arbeiten.
