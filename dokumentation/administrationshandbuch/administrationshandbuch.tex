\section{Administrationshandbuch}
Hier wird mit Beispielbildern erklärt, wie die Software bedient wird und was der Benutzer alles 
in der Software ausführen kann.

\subsection{Chat}
Den Chat kann man mit localhost:4200/chat erreichen. Wenn localhost:4200
eingegeben wird, wird der Benutzer zu der Adresse vom chat weitergeleitet.
An der Seite angekommen kann der Nutzer dann mit dem ChatBot schreiben. 
Der Benutzer kann mit dem ChatBot interagieren indem er eine Frage in dem Textfeld schreibt
und dann mit dem Sendebutton versendet.

\subsection{Admininterface}
Das Admininterface ist nur für Admins erreichbar mit localhost:4200/admin-interface. Der Admin muss sich per Keycloak
anmelden und wird erst dann weitergeleitet zum Bearbeiten der verschiednen Funktionen
des Chatbots.
Nach dem Anmelden mit Kezcloak wird der Benutzer zur Infopage Seite weitergeleitet.
welche mit der Adresse localhost:4200/admin-interface/infopage gekennzeichnet ist.

\subsubsection{Allgemein}
Das Allgemein ist für den Admin erreichber mit localhost:4200/admin-interface/allgemein. 
Der Admin kann dort den Bot Avatar ändern diese wird dann auch im chat geändert.

\subsubsection{Korpus}
Der Korpus ist für den Admin erreichbar mit localhost:4200/admin-interface/corpus.]
Im Korpus kann der Admin einen neuen Intent hinzufügen und auch diese Entfernen. 
Er kann außerdem auch Fragen, die er den Bot stellen will erstellen und auch die 
dazugehörige Antwort hinzufügen.

\subsubsection{Einstellungen}
Der Einstellungsbereich ist für den Admin zugänglich mit der Adresse 
localhost:4200/admin-interface/einstellungen.
Hier kann der Admin seine Domänen zu den Gruppen hinzufügen. Die Korpusdaten, die zugeteilt sind,
können nur von der Gruppe benutzt werden, welche diese zugeteilt bekommen hat. 
Zum Beispiel kann ein nicht registrierter Nutzer nur die Basis Korpusdaten einsehen und nicht die von der Hochschule.