\section{Risikoanalyse}

Auf den folgenden Seiten haben wir in einer Tabelle möglicher Risiken für unser Projekt aufgelistet.
Sowie Maßnahmen wie wir diese Risiken verringern möchten. \\

\begin{table}[H]
    \begin{center}
        \begin{tabular}{|p{120px}|c|c|p{150px}|}
            \hline
            \begin{tabular}{l}
                Risiko \\
            \end{tabular}
             &
            \begin{tabular}{l}
                Eintritts-  \\
                wahrschein- \\
                lichkeit    \\
            \end{tabular}
             &
            \begin{tabular}{l}
                Auswirkung \\
            \end{tabular}
             &
            \begin{tabular}{l}
                Maßnahme \\
            \end{tabular}
            \\
            \hline
            \begin{tabular}{l}
                Team schafft es  \\
                nicht schnell    \\
                genug Typescript \\
                zu lernen        \\
            \end{tabular}
             &
            \begin{tabular}{l}
                niedrig \\
            \end{tabular}
             &
            \begin{tabular}{l}
                hoch \\
            \end{tabular}
             &
            \begin{tabular}{l}
                Alle Teammitglieder \\
                beginnen frühzeitig \\
                sich mit Typescript \\
                zu beschäftigen     \\
            \end{tabular}
            \\
            \hline
            \begin{tabular}{l}
                Team schafft es \\
                nicht Keycloak  \\
                zu integrieren  \\
            \end{tabular}
             &
            \begin{tabular}{l}
                niedrig \\
            \end{tabular}
             &
            \begin{tabular}{l}
                hoch \\
            \end{tabular}
             &
            \begin{tabular}{l}
                Alle Teammitglieder     \\
                schauen sich recht-     \\
                zeitig die Einführungs- \\
                videos von Herr Rößler  \\
                zu Keycloak an          \\
            \end{tabular}
            \\
            \hline
            \begin{tabular}{l}
                Das Team hat      \\
                Schwierigkeiten   \\
                das Userinterface \\
                in Angular zu     \\
                entwickeln        \\
            \end{tabular}
             &
            \begin{tabular}{l}
                niedrig \\
            \end{tabular}
             &
            \begin{tabular}{l}
                hoch \\
            \end{tabular}
             &
            \begin{tabular}{l}
                Das Team greift auf  \\
                gut bewährte Designs \\
                zurück               \\
            \end{tabular}
            \\
            \hline
            \begin{tabular}{l}
                Das Team hat    \\
                Schwierigkeiten \\
                eine Verbindung \\
                mit socket.io   \\
                herzustellen    \\
            \end{tabular}
             &
            \begin{tabular}{l}
                niedrig \\
            \end{tabular}
             &
            \begin{tabular}{l}
                hoch \\
            \end{tabular}
             &
            \begin{tabular}{l}
                Das Team informiert    \\
                sich rechtzeitig auf   \\
                der Socket.io Website, \\
                wie eine Verbindung    \\
                aufgebaut wird         \\
            \end{tabular}
            \\
            \hline
            \begin{tabular}{l}
                Das Admin Interface \\
                lässt sich nicht    \\
                flexibel genug      \\
                anpassen            \\
            \end{tabular}
             &
            \begin{tabular}{l}
                niedrig \\
            \end{tabular}
             &
            \begin{tabular}{l}
                niedrig \\
            \end{tabular}
             &
            \begin{tabular}{l}
                Das Team informiert     \\
                sich rechtzeitig        \\
                welche Möglichkeiten    \\
                es in das UI einbauen   \\
                möchte, um Flexibilität \\
                zu garantieren          \\
            \end{tabular}
            \\
            \hline
            \begin{tabular}{l}
                Der ChatBot     \\
                antwortet stark \\
                verzögert       \\
            \end{tabular}
             &
            \begin{tabular}{l}
                niedrig \\
            \end{tabular}
             &
            \begin{tabular}{l}
                mittel \\
            \end{tabular}
             &
            \begin{tabular}{l}
                Das Team muss mit-  \\
                einplanen, dass der \\
                ChatBot in kleine   \\
                saubere Module      \\
                aufgeteilt wird     \\
            \end{tabular}
            \\
            \hline
        \end{tabular}
    \end{center}
    \caption{Risikoanalyse Tabelle Teil 1}
    \label{tab:Risikoanalyse_Teil_1}
\end{table}
%Second part of the table
\begin{table}[H]
    \begin{center}
        \begin{tabular}{|p{120px}|c|c|p{150px}|}
                       \hline
            \begin{tabular}{l}
                Risiko \\
            \end{tabular}
             &
            \begin{tabular}{l}
                Eintritts-  \\
                wahrschein- \\
                lichkeit    \\
            \end{tabular}
             &
            \begin{tabular}{l}
                Auswirkung \\
            \end{tabular}
             &
            \begin{tabular}{l}
                Maßnahme \\
            \end{tabular}
            \\
            \hline
            \begin{tabular}{l}
                Hohe Latenz durch \\
                alle Komponenten  \\
            \end{tabular}
             &
            \begin{tabular}{l}
                unwahr-    \\
                scheinlich \\
            \end{tabular}
             &
            \begin{tabular}{l}
                mittel \\
            \end{tabular}
             &
            \begin{tabular}{l}
                Das Team muss den       \\
                ChatBot testen,         \\
                um z.B. Endlosschleifen \\
                zu verhindern           \\
            \end{tabular}
            \\
            \hline
            \begin{tabular}{l}
                Der ChatBot hat    \\
                Schwierigkeiten    \\
                Sätze zu verstehen \\
            \end{tabular}
             &
            \begin{tabular}{l}
                mittel \\
            \end{tabular}
             &
            \begin{tabular}{l}
                mittel \\
            \end{tabular}
             &
            \begin{tabular}{l}
                Das Team muss bei     \\
                einem Regex Ansatz    \\
                mehrere Regex Befehle \\
                vordefinieren, um ein \\
                großes Spektrum       \\
                abzudecken            \\
            \end{tabular}
            \\
            \hline
            \begin{tabular}{l}
                Die Hardware des \\
                Kunden ist nicht \\
                kompatibel mit   \\
                der Software     \\
            \end{tabular}
             &
            \begin{tabular}{l}
                niedrig \\
            \end{tabular}
             &
            \begin{tabular}{l}
                hoch \\
            \end{tabular}
             &
            \begin{tabular}{l}
                Das Team muss       \\
                frühzeitig mit dem  \\
                Kunden klären für   \\
                welche Hardware der \\
                ChatBot entwickelt  \\
                werden soll         \\
            \end{tabular}
            \\
            \hline
            \begin{tabular}{l}
                Das Team scheitert \\
                einen MVP zu       \\
                entwickeln         \\
            \end{tabular}
             &
            \begin{tabular}{l}
                niedrig \\
            \end{tabular}
             &
            \begin{tabular}{l}
                hoch \\
            \end{tabular}
             &
            \begin{tabular}{l}
                Das Team muss sehr    \\
                früh mit der Imple-   \\
                mentierung beginnen   \\
                und ausführlich genug \\
                recherchieren         \\
            \end{tabular}
            \\
            \hline
            \begin{tabular}{l}
                Das Team scheitert \\
                rechtzeitig genug  \\
                alle Technologien  \\
                zu lernen          \\
            \end{tabular}
             &
            \begin{tabular}{l}
                niedrig \\
            \end{tabular}
             &
            \begin{tabular}{l}
                hoch \\
            \end{tabular}
             &
            \begin{tabular}{l}
                Das Team recherchiert   \\
                frühzeitig und versucht \\
                für alle möglichen      \\
                Probleme Lösungen       \\
                zu finden               \\
            \end{tabular}
            \\
            \hline
        \end{tabular}
    \end{center}
    \caption{Risikoanalyse Tabelle Teil 2}
    \label{tab:Risikoanalyse_Teil_2}
\end{table}
\newpage
\noindent Fazit: \\
Bei Beginn unseres Projektes haben wir mögliche Risiken aufgelistet,
die in unserem Projekt auftauchen könnten
und wie wir diese nach Möglichkeit verhindern möchten.
Wir haben dabei herausgefunden, dass wir sehr viele Risiken haben,
die von niedriger Eintrittswahrscheinlichkeit sind.
Dennoch sind die Mehrheit der Risiken von hoher Auswirkung im Projekt.
Als Beispiel, das Team schafft es nicht schnell genug Typescript zu lernen.
Die wichtigsten Risiken für uns waren die oben genannten Risiken zu Angular, Keycloak und Socket.io.
Wenn eines dieser Risiken von unserem Team nicht ausgleichbar wäre,
dann würden wir gar nicht die Möglichkeit haben einen MVP oder
später ein fertiges funktionierendes Produkt abzuliefern.
Mit den Tabellen \ref{tab:Risikoanalyse_Teil_1} und \ref{tab:Risikoanalyse_Teil_2}  möchten wir für unsere Gruppe festhalten,
welche Gedanken wir uns über die möglichen Risiken in unserem Projekt gemacht haben.
Damit wir besser und effizienter unser Projekt vorantreiben können und
vorbereitet sind auf mögliche Schwierigkeiten.