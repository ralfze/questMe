\section{Aufteilung des Teams}
Hier werden die einzelnen Aufgaben, die wir bearbeitet haben in Issues mit dem Namen der Beteilligten aufgezählt.

\subsection{Herr Ralf Zeller}
Hier werden die Issues von Herr Ralf Zeller aufgelistet.
\begin{enumerate}
    \item Angular vs. React.js vs, Vue.js recherchiert
    \item Videoinhalt für die Zwischenpräsentation erstellt
    \item Die Zwischenpräsentationsfolien erstellt
    \item Latex Grundstruktur der Dokumentation erstellt
    \item Die Meilensteine in der Dokumentation erstellt
    \item Die Risk Liste erstellt
    \item Technologien in der Dokumentation eingetragen
    \item Die Commits und die Labeling bestimmen (in welcher Sprache und in welcher Form diese geschrieben werden)
    \item Die Datenbank für den Korpus erstellt
    \item Ausgetestet ob NLP mit mongoDB benutzt werden kann
    \item Datenbankstruktur (ERM) erstellt
    \item Die Technologien in der Dokumentation aktualisiert
    \item Den SRC von der "Literatur.bib" geprüft
    \item Risikoanalyse geprüft und ein Review geschrieben
    \item Meilensteine korrigiert
    \item Die Technologien geprüft und die Liste neu gestaltet
    \item Die Use Cases enummeriert
    \item Die Dokumentation Tree im Gitlab gesäubert
    \item Das UML Komponentendiagramm hinzugefügt
    \item Den NLP Node.js Server von  14+ zu 16+ aktualisiert
    \item Die Dokumentation korrigiert
    \item Die Admin development Branch gesäubert
    \item Docker Compose erstellt
    \item NLP Test Branch und presentation Branch rebased
    \item Chat/Admin UI, NLP und mongoDb kombiniert
    \item Admin Interface in Docker Compose integriert
    \item Funktionalität zu Admin Corpus Website hinzugefügt
    \item Keycloak in Docker Compose integriert
    \item Erste Version von der Dokumentation für die Abgabe hinzugefügt
    \item Versionen von allen Technologien überprüft
    \item NPM update am 11.01.22 durchgeführt
    \item Team Building Retrospective gehalten
    \item Corpus for Interna erstellt
    \item Die main Branch gesäubert
    \item Hinzufüge-/ und Entfern-/ Button für die Intent Cards im Admin Interface implementiert
    \item Die essentiellen Funktionen für die Rest API implementiert
    \item Allgemein und Einstellungen im Admin Interface bearbeitet
    \item Verbindung zwischen der Allgemein(General) Seite und dem Backend implementiert
    \item Verbindung zwischen Einstellungen(Settings) Seite und dem Backend implementiert
    \item JSON als Korpus in mongoDb integriert
    \item Den Hinzufüge Button auf der Korpus Seite ändern und den Entfernen Button integriert
    \item Installations- und Administrationshandbuch hinzugefügt
    \item Ausblick ausgedacht
    \item Fix Intent Cards in Container HTML intent-array Intent Karten in Container HTML intent-array fixiert
    \item Nicht benutzte Branches gelöscht
    \item Readme über Gitlab clean hinzugefügt
    \item Den Icon von der Katze im Chat Interface geändert
    \item READEME für questME Gitlab Repository hinzugefügt
    \item Die Technologien in der Dokumentation aktualisiert
    \item Alle Meeting Protokolle verfasst
    \item Alle Meeting Protokolle in die Dokumentation Branch gepusht
    \item UI in Englisch umgeschrieben
    \item Kapitel zu den Docker Images erstellt
\end{enumerate}

\subsection{Frau Pavithra Sureshkumar}
Hier werden die Issues von Frau Pavithra Sureshkumar aufgelistet.
\begin{enumerate}
    \item Am Anfang des Projekts: Aufgabenverteilung und Milestones mit Prioritäten erstellen
    \item Angular vs. React.js vs, Vue.js recherchiert
    \item Videoinhalt für die Zwischenpräsentation erstellt
    \item Die Zwischenpräsentationsfolien erstellt
    \item UI Designs erstellt und dokumentiert
    \item Use Cases erstellt und dokumentiert
    \item About Us hinzugefügt
    \item User Stories erstellt und dokumentiert
    \item Angular getestet und Chat Interface erstellt
    \item User stories neu formuliert
    \item getrennte Komponentendiagramme erstellt
    \item Kriterien für Usability erstellt
    \item Zielgruppe, Problem, Eigenschaften, Alleinstellungsmerkmal erstellt
    \item Merge fault von User Stories korrigiert
    \item Admin Interface mit Angular erstellt
    \item An Corpus gearbeitet
    \item Usability-Test erstellt und hinzugefügt
    \item UI Designs korrigiert
    \item Appendix erstellt
    \item Die Dokumentation Tree im Gitlab gesäubert
    \item User Story mit Acceptance Criteria erstellt und dokumentiert
    \item Den UML Komponentendiagramm hinzugefügt
    \item Die Admin development Branch gesäubert
    \item Funktionalität zu Admin Corpus Website hinzugefügt
    \item UI Designs Version 1/2 korrigiert
    \item Team Building Retrospective gehalten
    \item Corpus for University erstellt
    \item Corpus for Interna erstellt
    \item Angular CI/CD for automated testing recherchiert (aber keine Zeit gehabt auszutesten)
    \item Hinzufüge-/ und Entfern-/ Button für die Intent Cards im Admin Interface implementiert
    \item Allgemein und Einstellungen im Admin Interface gearbeitet
    \item Chat/Admin UI, NLP und mongoDb kombiniert
    \item Verbindung zwischen der Allgemein(General) Seite und Backend implementiert
    \item Verbindung zwischen Einstellungen(Settings) Seite und Backend implementiert
    \item Ein Hintergrundbild für den Chat hinzugefügt
    \item Auswahl an Hintergründen für den Chat erstellt
    \item Angefangen an der Enddokumentation zu arbeiten
    \item Installations- und Administrationshandbuch hinzugefügt
    \item Die Aufgabenverteilung von Ralf und Pavithra dokumentiert
    \item Die Reflektion vom Projektmanagement hinzugefügt
    \item Benutzte Lizenzen und Projekt Lizenzen eingetragen
    \item Ausblick ausgedacht und dokumentiert
    \item Team Reflektion vom Lernfortschritt verfasst und dokumentiert
    \item Angefangen die Endpräsentation vorzubereiten und Präsentationsfolien erstellt
    \item Endpräsentation korrigiert und bearbeitet
    \item Alle Meeting Protokolle verfasst
    \item Meeting Protokolle als PDF in Appendix (in Latex) hinzugefügt
    \item Aufgabenverteilung korrigiert
\end{enumerate}

\subsection{Herr Kevin Sautner}
Hier werden die Issues von Herr Kevin Sautner aufgelistet.
\begin{enumerate}
    \item Recherche zu MongoDB vs. PostgreSQL
    \item Tabellen MongoDB vs. PostgreSQL und Angular vs. Vue.js hinzugefügt
    \item Gantt-Diagramm überarbeitet und hinzugefügt
    \item UML-Verteilungsdiagramm erstellt und hinzugefügt
    \item Fazit zu den Technologien-Tabellen hinzugefügt
    \item Tests mit Keycloak durchgeführt
    \item Gantt-Diagramm um Milestones erweitert
    \item Zielgruppe Professor erstellt
    \item Beschreibung zum Gantt-Diagramm hinzugefügt
    \item Keycloak implementiert
    \item Gantt- und Verteilungsdiagramm überarbeitet
    \item Fazit zu Tabellen überarbeitet
    \item Teile der Use Cases und User Stories erstellen und überarbeitet
    \item Unterpunkte von Technologien neu angeordnet
    \item Rechtschreibprüfung der Dokumentation zur Zwischenabgabe
    \item Versucht Keycloak auf SAML umzustellen
    \item Grundstruktur der Rest-API erweitert
    \item Keycloak Authentifizierung in die Rest-API implementiert
    \item Auslesen und Senden der Nutzerrollen mit einer Chat-Nachricht
    \item Kapitel zur Sicherheit hinzugefügt
    \item UML-Verteilungsdiagramm angepasst
    \item Keycloak Administrationshandbuch hinzugefügt
\end{enumerate}