\section{Aufteilung des Teams}
Hier werden die einzelnen Aufgaben, die wir bearbeitet haben in Issues mit dem Namen der Beteilligten aufgezählt.

\subsection{Herr Ralf Zeller}
Hier werden die Issues von Herr Ralf Zeller aufgelistet.
\begin{enumerate}
\item Angular vs. React.js vs, Vue.js recherchiert
\item Die Zwischenpräsentationsfolien erstellt
\item Latex Grundstruktur der Dokumentation erstellt
\item Die Meilensteine in der Dokumentation erstellt
\item Die Risk Liste erstellt
\item Technologien in der Dokumentation eingetragen
\item Die Commits und die Labeling bestimmen (in welcher Sprache und in welcher Form diese geschrieben werden)
\item Die Datenbank für den Korpus erstellt
\item Ausgetestet ob NLP mit mongoDB benutzt werden kann 
\item Datenbankstruktur (ERM) erstellt
\item Die Technologien in der Dokumentation aktualisiert
\item Den SRC von der "Literatur.bib" geprüft
\item Risikoanalyse geprüft und ein Review geschrieben
\item Meilensteine korrigiert
\item Die Technologiengeprüft und die Liste neu gestaltet
\item Die Use Cases enummeriert
\item Die Dokumentation Tree im Gitlab gesäubert
\item Den UML Komponentendiagramm hinzugefügt
\item Den NLP Node.js Server von  14+ zu 16+ aktualisiert
\item Die Dokumentation korriegiert
\item Die Admin development branch gesäubert
\item Den ersten Commit umbennant
\item Docker Compose erstellt
\item NLP Test branch und presentation branch rebased
\item Chat/Admin UI, NLP und mongoDb kombiniert
\item Admin Interface in Docker Compose integriert
\item Funktionalität zu Admin Corpus Website hinzugefügt
\item Keycloak in Docker Compose integriert
\item Erste Version von der Dokumentation für die Abgabe hinzugefügt
\item Vesionen von allen Technologien überprüft
\item Team Building Retrospective gehalten
\item Corpus for Interna erstellt
\item Die main branch gesäubert
\item Hinzufüge-/ und Entfern-/ Button für die Intent Cards im Admin Interface implementiert
\item Die essentielle Funktion für die Rest API implementiert
\item Allgemein und Einstellungen im Admin Interface gearbeitet
\item Verbindung zwischen der Allgemein Seite und Backend implementiert
\item Verbindung zwischen Einstellungen Seite und Backend implementiert
\item JSON als Korpus in mongoDb integriert
\item Den Hinzufüge Button auf der Korpus Seite ändern und den Entfern Button integriert
\item Installations- und Administrationshandbuch hinzugefügt
\item Benutzte licences und project licence eingetragen
\item Ausblick ausgedacht
\item Fix Intent Cards in container HTML intent-array Intent Karten in Container HTML intent-array fixiert
\item Nicht benutzte Branches gelöscht
\item Readme über Gitlab clean hinzugefügt
\item Den Icon von der Katze im Chat Interface geändert
\item READE.ME für questME Gitlab Repository hinzugefügt
\item Die Technologien in der Dokumentation aktualisiert
\item Alle Meeting Protokolle verfasst
\item Alle Meeting Protokolle in die Dokumentation branch gepusht
\item UI in englisch umschreiben
\end{enumerate}

\subsection{Frau Pavithra Sureshkumar}
Hier werden die Issues von Frau Pavithra Sureshkumar aufgelistet.
\begin{enumerate}
    \item Am Anfang des Projekts: Aufgabenverteilung und Milestones mit Prioritäten erstellen 
    \item Angular vs. React.js vs, Vue.js recherchiert
    \item Die Zwischenpräsentationsfolien erstellt
    \item UI Designs erstellt und dokumentiert
    \item About Us hinzugefügt
    \item User Stories erstellt und dokumentiert
    \item Angular getestet und Chat Interface erstellt
    \item User stories neu formuliert
    \item getrennte Komponentendiagramm nerstellt
    \item Kriterien für Usability erstellt
    \item Zielgruppe, Problem, Eigenschaften, Alleinstellungsmerkmal erstellt
    \item merge fault von User Stories korrigiert
    \item Admin Interface mit Angular erstellt
    \item An Corpus gearbeitet
    \item Usability-Test erstellt und hinzugefügt
    \item UI designs korrigiert
    \item Appendix erstellt
    \item Die Dokumentation Tree im Gitlab gesäubert
    \item User Story mit Acceptance Criteria erstellt und dokumentiert
    \item Den UML Komponentendiagramm hinzugefügt
    \item Die Admin development branch gesäubert
    \item Funktionalität zu Admin Corpus Website hinzugefügt
    \item UI Designs Version1/2 korrigiert 
    \item Team Building Retrospective gehalten
    \item Corpus for University erstellt
    \item Corpus for Interna erstellt
    \item Angular CI/CD for automated testing recherchiert (aber keine Zeit gehabt auszutesten)
    \item Hinzufüge-/ und Entfern-/ Button für die Intent Cards im Admin Interface implementiert
    \item Allgemein und Einstellungen im Admin Interface gearbeitet
    \item Chat/Admin UI, NLP und mongoDb kombiniert
    \item Verbindung zwischen der Allgemein Seite und Backend implementiert
    \item Verbindung zwischen Einstellungen Seite und Backend implementiert
    \item Ein Hintergrundbild für den Chat hinzugefügt
    \item Angefangen an der Enddokumentation zu arbeiten
    \item Installations- und Administrationshandbuch hinzugefügt
    \item Die Aufgabenverteilung von Ralf und Pavithra dokumentiert
    \item Die Reflektion vom Projektmanagement hinzugefügt
    \item Benutzte licences und project licence eingetragen
    \item Ausblick ausgedacht und dokumentiert
    \item Team Reflektion vom Lernfortschritt verfasst und dokumentiert
    \item Angefangen die Endpräsentation vorzubereiten und Präsentationsfolien erstellt
    \item Alle Meeting Protokolle verfasst
    \item Meeting Protokolle als PDF in Appendix (in Latex) hinzugefügt
    \item Aufgabenverteilung korrigiert
\end{enumerate}