\section{Use Cases}
Hier werden wir unsere User Stories zu den einzelnen Zielgruppen auflisten.
Wir haben die Zielgruppen nicht registrierte Nutzer, Professor und Student.

\subsection{Struktur der Use Cases}
\textbf{Use Case:}
\newline

\noindent Als \textbf{<Akteur>} möchte ich \textbf{<Funktion>}, um \textbf{<Nutzen>} zu erreichen.
\newline

\noindent \textbf{Akzeptanzkriteria:}
\newline

\noindent Szenario: 

\noindent kurze Beschreibung des Szenarios
\newline

\noindent Wenn ich \dots

\noindent und \dots

\noindent Dann \dots

\subsection{Student}

\textbf{Use Case:}
\newline

\noindent Als Student möchte ich meine Informationen von meinem Stundenplan abrufen können,
um meine Vorlesungen einsehen zu können.
\newline

\noindent \textbf{Akzeptanzkriteria:}
\newline

\noindent Szenario: 

\noindent Der angemeldete Student fragt nach, wann die Vorlesung Virtuelle Realität stattfindet.
\newline

\noindent Wenn ich nach der Uhrzeit von der Vorlesung Virtuelle Realität frage

\noindent und mich im Chatfenster befinde.

\noindent Dann bekomme ich meine Vorlesungsinformationen zu Virtuelle Realität.

\newpage
\subsection{Unregistrierter Nutzer}

\textbf{Use Case:}
\newline

\noindent Als unregistrierter Nutzer,
möchte ich Zugang zur Chatbot Chat Seite haben,
damit ich einen Smalltalk mit dem Chatbot führen kann.
\newline

\noindent \textbf{Akzeptanzkriteria:}
\newline

\noindent Szenario: 

\noindent Als unregistrierter Nutzer,
möchte ich Zugang zur Chatbot Chat Seite haben,
damit ich einen Smalltalk mit dem Chatbot führen kann.
\newline

\noindent Wenn ich ein unregistrierter Nutzer bin und mich auf der Chatbot Chatseite befinde

\noindent und in das Chatfenster Fragen schreibe und diese abschicke.

\noindent Dann antwortet der Chatbot auf meine Fragen mit allgemeinen Antworten.

\subsection{Professor (Admin)}

\textbf{Use Case:}
\newline

\noindent Als Professor möchte ich den Korpus des Chatbots 
um eine Frage und Antwort Möglichkeit erweitern, 
um den Nutzern ein größeres Repertoire zu bieten.
\newline

\noindent \textbf{Akzeptanzkriteria:}
\newline

\noindent Szenario: 

\noindent Der angemeldete Professor möchte eine neue Frage-Antwort Möglichkeit hinzufügen.
\newline

\noindent Wenn ich mich auf der Korpus-Seite des Admin-Interfaces befinde

\noindent und auf den Button zum Hinzufügen einer Frage-Antwort Möglichkeit drücke.

\noindent Dann kann ich eine neue Frage und eine dazugehörige Antwort eintippen.




