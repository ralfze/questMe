\section{Zielgruppen}
In diesem Abschnitt Listen wir unsere drei Zielgruppen: Nicht registrierte Nutzer, Student und Professor.
Zu den Zielgruppen werden wir jeweils, ihre Probleme, was ihre geforderten Eigenschaften sind beschreiben und was bei unserer Software für Sie 
einzigartig ist.


\subsection{Zielgruppe: Nicht registrierte Nutzer}

\textbf{Probleme:}
\begin{itemize}
    \item Interesse an einem Smalltalk mit einem Chatbot
    \item Möchten einfache Fragen beantwortet bekommen
    \item Haben bisher mit sehr monoton antworteten Chatbots gechattet
\end{itemize}
\medskip

\textbf{Eigenschaften}
\begin{itemize}
    \item Nicht unbedingt technikaffin
    \item Möchten eine leichte Bedienung
    \item Haben sehr wenig Erfahrung mit Chatbots
    \item Interesse an einem Smalltalk
    \item Möchten unterhalten werden
    \item Kennen WhatsApp, Skype, etc.
\end{itemize}
\medskip

\textbf{Alleinstellungsmerkmal / Einzigartigkeit:}
\begin{itemize}
    \item Möchten sich sehr schnell in die Chatoberfläche einfinden
    \item Möchte ein einfaches und simples Userinterface
    \item Möchten, dass der Chatbot wie ein Mensch mit ihnen chattet
\end{itemize}

\newpage
\subsection{Zielgruppe: Student}

\textbf{Probleme:}
\begin{itemize}
    \item Studenten finden ihre Raumnummer nicht mehr
    \item Studenten wissen nicht, wann Ihre Vorlesung stattfindet
    \item Studenten müssen immer ihren Stundenplan einsehen und verlieren dabei Zeit 
    \item Studenten müssen immer manuell nach Informationen zu ihren Kursen suchen
\end{itemize}
\medskip

\textbf{Eigenschaften}
\begin{itemize}
    \item Studenten möchten alles einfacher
    \item Keine komplizierten Umwege
    \item Einfache Bedienung
    \item Gutes bzw. ansprechendes Design
    \item Gezielte Antworten auf bestimmte Fragen
\end{itemize}
\medskip

\textbf{Alleinstellungsmerkmal / Einzigartigkeit:}
\begin{itemize}
    \item Studenten können mehrere Fragen stellen
    \item Eine leichte, aber ansprechende Bedienung des Chats
    \item Bekannte Chatoberfläche
    \item Keine komplizierten Umwege 
\end{itemize}

\newpage
\subsection{Zielgruppe: Professor}

\textbf{Probleme:}
\begin{itemize}
    \item Professoren finden ihre Raumnummer nicht mehr
    \item Professoren wissen nicht, wann Ihre Vorlesung stattfindet
    \item Professoren müssen immer ihren Stundenplan einsehen und verlieren dabei Zeit
    \item Professoren wissen nicht, ob ein Raum frei ist, wenn sie sich z.B. einen größeren suchen müssen
    \item Professoren wissen nicht, wann der Prüfungstermin für ihre Vorlesung ist
    \item Professoren wollen ihren Studenten Informationen übermitteln
    \item Professoren wollen ihren Studenten bei Problemen helfen
\end{itemize}
\medskip

\textbf{Eigenschaften}
\begin{itemize}
    \item Professoren möchten alles einfacher
    \item Keine komplizierten Umwege
    \item Einfache Bedienung
    \item Gutes bzw. ansprechendes Design
    \item Gezielte Antworten auf bestimmte Fragen
\end{itemize}
\medskip

\textbf{Alleinstellungsmerkmal / Einzigartigkeit:}
\begin{itemize}
    \item Professoren können mehrere Fragen stellen
    \item Eine leichte, aber ansprechende Bedienung des Chats
    \item Bekannte Chatoberfläche
    \item Keine komplizierten Umwege
    \item Professoren übernehmen die Rolle des Admins
    \item Professoren können die Einstellungen des Chatbots verwalten
    \item Professoren können den Korpus um Fragen und Antworten erweitern
\end{itemize}



