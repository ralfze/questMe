\subsection{Docker Images}
Unsere Anwendung besteht aus vier Docker Images,
die aus verschiedenen Komponenten aufgebaut sind.
Bei der Basis der Docker Images haben wir die nach Möglichkeit die Linux Alpine Version 3.14 verwendet.
Dadurch war es uns möglich Images von einer kleineren Speichergröße zu erstellen.

\subsubsection{Angular-Frontend}
\begin{itemize}
    \item Basis: Alpine Version 3.14
    \item Nodejs Version: 16.13.0
    \item Angular Port: 4200
\end{itemize}

\subsubsection{NodeWebApp}
\begin{itemize}
    \item Basis: Alpine Version 3.14
    \item Nodejs Version: 16.13.0
    \item Chatbot Port: 3000
    \item REST Port: 3001
\end{itemize}

\subsubsection{mongoDb}
\begin{itemize}
    \item Basis: Ubuntu focal mit mongoDb 5.0.4
    \item mongoDb Port: 27017
\end{itemize}

\subsubsection{Keycloak}
\begin{itemize}
    \item Basis: Alpine Version 3.14
    \item Java: Openjdk Version 11
    \item Keycloak Port: 8080
\end{itemize}


