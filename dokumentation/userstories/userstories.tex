\section{User Stories}
In diesem Kapitel haben wir unsere User Stories gesammelt.

\subsection{Struktur}
Um \textbf{<Nutzen>} möchte ich, als \textbf{<Akteur>, <Funktion>}.
\subsection{User Stories Version 1}
Hier listen wir unsere ersten Ideen auf, die wir erfüllen möchten.
\\

\textbf{Chatfenster}
\begin{enumerate}
    \item Um eine Nachricht abzuschicken möchte ich, als Nutzer, einen Sendbutton betätigen.
    \item Um eine Antwort zu erhalten möchte ich, als Nutzer, in ein Eingabefeld meine Frage stellen.
    \item Um eine Nachricht zu erhalten möchte ich, als Nutzer, ein CHatfenster haben.
    \item Um meine Nachrichten einzusehen möchte ich, als Nutzer, einen Scrollbarren haben.
    \item Um die Nachricht des Bots von meiner Nachricht zu unterscheiden möchte ich, als Nutzer, das Icon von dem Bot sehen.
    \item Um die Nachricht des Bots von meiner Nachricht zu unterscheiden möchte ich, als Nutzer, verschiedene Farben für die Sprechblasen haben.
    \item Um eine Nachricht zu Schreiben möchte ich, als Nutzer, hingewiesen werden, wo ich Schreiben soll.
\end{enumerate}

\textbf{Admin Interface: Allgemein}
\begin{enumerate}
    \item Um auf Allgemein zu kommen möchte ich, als Admin, ein dropdown Menü oder etwas gleichwertiges haben.
    \item Um auf Allgemein mein Bot Avatar zu wecheln möchte ich, als Admin, verschiedene Icons zur Auswahl haben.
    \item Um zu erkennen welchen Bot Avatar ich gewählt habe möchte ich, als Admin, einen Haken als ausgewählt gezeigt bekommen.
    \item Um mein ChatBot Namen zu editieren möchte ich, als Admin, ein Editierbutton haben.
    \item Um mein ChatBot Namen einzugeben möchte ich, als Admin, ein Eingabefeld benutzen können.
    \item Um ein Eingabefeld zu erkennen möchte ich, als Admin, eine gefärbte Fläche sehen.
    \item Um zu Erkennen ob ich im Allgemein bin, möchte ich, als Admin, die ausgewählte Fläche in einer anderen Farbe sehen.
\end{enumerate}

\textbf{KeyCloak}
\begin{enumerate}
    \item Um KeyCloak Einstellungen zu verwalten, möchte ich, als Admin, mich in KeyCloak einloggen.
    \item Um die Einstellungen des Chatbots zu verwalten, möchte ich, als Admin, mich in das Admin-Interface einloggen.
    \item Um relevante Daten mitzuteilen, möchte ich, als Hochschulangehöriger, mich mit dem Shibboleth SSO der Hochschule einloggen.
    \item Um Zeit zu sparen, möchte ich, als Admin, schnellen Zugriff auf das KeyCloak-Webinterface über das Admin-Interface.
    \item Um Zielgerichteter Fragen beantworten zu können, möchte ich, als Admin, neue Nutzergruppen erstellen.
    \item Um die möglichen Fragen einer Gruppe zu erweitern, möchte ich, als Admin, einer Nutzergruppe einen neuen Fragensatz zuweisen.
    \item Um mögliche Fragen für eine Gruppe einzuschränken, möchte ich, als Admin, einen, zu einer Nutzergruppe zugewiesenen, Fragensatz entfernen.
    \item Um bei Bedarf Änderungen vorzunehmen, möchte ich, als Admin, Nutzer in KeyCloak verwalten.
    \item Um sie optisch nach meinen Vorstellungen zu ändern, möchte ich, als Admin, die Login-Seite anpassen.
\end{enumerate}

\textbf{Node.js Allgemein}
\begin{enumerate}
    \item Um direkt mit dem ChatBot kommunizieren zu können möchte ich als Nutzer, eine bidirektionale Kommunikation zwischen dem Client und Server.
    \item Um mit dem ChatBot nach Kontext zu chatten möchte ich als Nutzer, dass der ChatBot den geschriebenen Kontext versteht.
    \item Um den Korpus des ChatBots zu speichern möchte ich als Admin, eine Möglichkeit den Korpus persistent zu speichern.
    \item Um mit dem ChatBot online zu kommunizieren möchte ich als Nutzer, dass der Bot über eine Webadresse erreichbar ist.
\end{enumerate}

\textbf{Admin Interface: Korpus}
\begin{enumerate}
    \item Um einen Überblick über den Korpus zu haben möchte ich, als Admin, eine Liste mit allen Fragen und Antworten.
    \item Um die Domaine wechseln zu können möchte, ich als Admin, ein Dropdown Menü.
    \item Um neue Fragen und Antworten hinzuzufügen möchte ich, als Admin, einen Button der einen neuen Eintrag hinzufügt.
    \item Um Fragen hinzuzufügen möchte ich, als Admin, eine Möglichkeit zum Hinzufügen.
    \item Um Antworten hinzuzufügen möchte ich, als Admin, eine Möglichkeit zum Hinzufügen.
    \item Um Fragen zu entfernen möchte ich, als Admin, eine Möglichkeit zum Entfernen.
    \item Um Antworten entfernen möchte ich, als Admin, eine Möglichkeit zum Entfernen.
    \item Um den Eintrag „Fragen und Antworten“ zu Bearbeiten möchte ich, als Admin, eine Möglichkeit den Eintrag zu bearbeiten.
\end{enumerate}

