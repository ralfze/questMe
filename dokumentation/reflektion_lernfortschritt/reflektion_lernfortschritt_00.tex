\section{Reflektion Lernfortschritt}
Hier wird der Lernforschritt, den wir als Gruppe gesammelt haben beschrieben.

\subsection{Reflektion Lernfortschritt von Frau Pavithra Sureshkumar}
Hier wird der Lernfortschritt von Frau Pavithra Sureshkumar zusammengefasst. Dabei werden nicht nur die technische Lernfortschritte, sondern auch die Teamarbeit als Reflektion erläutert.\newline

\noindent In unserem Projekt habe ich sehr viel Neues gelernt. Meine Aufgabe war es im Team das Frontend mit Angular zu programmieren und aufzusetzen.
Da ich kaum wissen über Typescript oder Angular hatte, musste ich mich erst mal damit beschäftigen, wie ich diese einsetzen kann. Dafür habe ich mich mit der 
Angular docs Seite beschäftigt \href{https://angular.io/docs}{https://angular.io/docs}, auch noch weitere Tutorials angeschaut und natürlich auch bei Problemen in Stackoverflow nachgeschaut. 
Als Richtlinie für die Komponenten habe ich Komponenten des Angular Material Bibliothek verwendet. 
Die Komponenten habe ich abgeändert und weitgehend nach unserem geplanten Prototyp realisiert \href{https://material.angular.io/}{https://material.angular.io/}. 
Ich habe gleich in der ersten Woche versucht das Chatinterface während dem Lernen zu programmieren.
Ich habe im Projekt sehr viel mit Ralf gearbeitet, weil ich mit ihm das Chat Interface, das ich programmiert habe mit dem Backend verbinden musste. Ich habe auch das Angular Routing angewandt um durch die ganzen
Seiten zu navigieren. Nachdem der Chat mit dem Backend verbunden wurde habe ich mich mit dem Admin Interface befasst. Dort musste ich das Routing einsetzen und die ganzen Seiten verbinden.
Ich habe mich mit Kevin ausgetauscht, wie es für ihn einfacher wäre auf die Routen zuzugreifen und habe diese auch so umgesetzt, wie er es haben wollte.
Nachdem ich die Komponenten mit Angular für das Admin Interface programmiert habe, konnte ich mein Wissen mit Ralf teilen. 
Dadurch ihn Unterstützen und dabei Helfen das Backend mit dem Frontend der Admin Interface Seiten zu verbinden, um den Seiten Funktionen zu geben.
Dabei haben wir beide gelernt, dass es viel einfacher gewesen wäre ein Datenmodell von vorne herein zu entwickeln. 
Ralf und ich haben es aber trotzdem geschafft das Chat- und das Admin- Interface so weit wie möglich zu programmieren.
Ich habe mich kurz mit dem CI/CD beschäftigt konnte diese aber nicht durchführen, weil ich andere Aufgaben erledigen musste und kaum Zeit gefunden habe.
Trotzdem habe ich mit Ralf gelernt, wie man eine Branch im Gitlab cleaned und wie man mit den erstellten Issues arbeitet.
Alles was ich gemacht und gelernt habe, habe ich in Gitlab in den Issues ausführlich dokumentiert. 
Ich habe mich nicht nur mit dem programmieren beschäftigt, sondern habe auch Richtlinien für das Dokumentieren erstellt und das meinen Teammitgliedern weitergegeben. 
Während ich programmiert habe, habe ich weiterhin an der Dokumentation gearbeitet.
Bei der Zwischenpräsentation und bei der Endpräsentation habe ich den ersten Schritt gewagt für die Gruppe die Präsentationsfolien zu erstellen und die allgemeine Gliederung zu erstellen.
Die Enddokumentation habe ich auch selber angefangen und habe mich dann mit Ralf ausgetauscht. 
Da Kevin sich nach den Weihnachtsferien nicht gemeldet hat. Habe ich seine Aufgaben ausgelassen und auf eine Ergänzung gewartet.\newline

\noindent Ich habe nicht nur fachliches Wissen erweitert und eingesetzt, sondern auch Teambildung und Teamarbeit ausgeführt. 
Wir haben trotz dass wir nur drei Personen sind, versucht Scrum auszuführen.
Das ein Team nicht immer einwandfrei läuft ist selbstverständlich. 
Es können immer Probleme bei Kommunikation entstehen und man könnte etwas falsch verstehen oder übermitteln. 
Dafür haben wir eine Retrospektive gemacht. Wir haben unsere Probleme angesprochen und eine Lösung gefunden. 
Trotzdem hat sich das Arbeitsverhalten von Kevin nicht geändert.
Auch haben wir gelernt, dass manche Personen nicht die Motivation haben etwas im Team zu leisten. 
Ralf und ich haben beschlossen dann etwas als Team zu unternehmen, um den Teamgeist zu steigern. 
Mit unserem Betreuer haben wir versucht jede Woche eine Besprechung zu halten und haben diese auch immer dokumentiert. 
Anschließend haben wir in einem Protokoll erstellt, dass die Besprechung zusammengfasst und es allen Meeting Mitglieder geschickt, um den Überblick zu behalten. 
Auch wenn man sich austauscht, sollte in jedem Team das selbstständige Arbeiten nicht ausfallen. 
Mit dem selbstständigen Arbeiten ist gemeint, dass man Aufgaben übernimmt und nicht nur die Aufgaben zugeteilt bekommt, das heißt Eigeninitiative sollte bei jedem vorhanden sein. 
Diese erwies sich schwer, weil es immer verschiedene Arbeitscharaktere gibt. 
Manche Personen haben sehr große Schwierigkeiten sich zu organisieren, selbstständig Aufgaben zu bearbeiten oder die nötige Motivation zu finden. 
Das alles haben wir im Seminar gelernt und haben eine große Lehre daraus gezogen.\newline
Außerdem haben wir die Planung unterschätzt. Wir hätten strengere Regeln setzen müssen, damit die Aufgabenverteilung gerecht verteilt wird. 
Da Kevin keine Eigeninitiative gezeigt hat mussten ich und Ralf ihm immer Aufgaben zuweisen. 
Am Ende hat er diese nicht mal ausgeführt, sich erst später damit beschäftigt und sich nicht mal damit auseinandergestzt, wie es für eine gute Bearbeitung des Projektes nötig gewesen wäre.
Ralf und ich hatten in der Gruppe mehr machen müssen und haben auf Hilfe von Kevin gewartet, die nie ankam. 
Am Anfang wurde die Planung so ausgeführt, dass sich jeder mit seinem Thema beschäftigen konnte und
relativ viel Zeit hatte alle Aspekte die wichtig sind zu recherchieren, um ein gutes Ergebnis für das Projekt zu erreichen.
Leider hat Kevin diese Aufgabe nicht ernst genommen oder halbherzig ausgeführt und es kam dazu, dass sich ein paar Gruppenmitglieder mit sehr viel Arbeit überbelastet haben. 
Gegen Ende haben wir noch versucht die Aufgabenverteilung gerechter zu verteilen. 
Wegen der ungerechten Aufgabenverteilung mussten Ralf und ich mehr Aufwand in das Projekt stecken, dass wir mental nicht mehr imstande waren auf Kevin aufzupassen. 
Währende den Ferien hat sich Kevin komplett von unserem Projekt abgekapselt und sich nicht mehr gemeldet.
Wir haben beschlossen, dass wir nicht seine Aufsichtsperson sind und uns um die Bearbeitung seiner Aufgaben kümmern müssen. 
Es ist nicht selbstverständlich in einer Gruppenarbeit einer Person ständig hinterher laufen zu müssen,
wenn jedes Gruppenmitglied das gleiche Arbeitspensum leisten muss. Damit ein gutes Ergebnis erreicht werden kann.
Ich habe dabei nur gelernt, dass ich dem Betreuer das Verhalten sehr früh weitergeben werde und nicht erst selbst Versuche das Problem zu lösen.
Ich sehe dieses Projekt als eine große Lehre und werde in den nächsten Projekten, die noch vor mir stehen meine Erfahrungen die aus diesem Projekt hervorgegangen sind Einsetzen.
Trotz aller Schwierigkeiten und Probleme hatte ich dennoch Spaß an den Aufgaben des Projektes und den neuen Erfahrungen die ich gesammelt habe.