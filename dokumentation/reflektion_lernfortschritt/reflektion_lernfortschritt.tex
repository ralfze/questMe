\section{Reflektion Lernfortschritt}
Hier wird der Lernforschritt, den wir als Gruppe gesammelt haben beschrieben.

\subsection{Reflektion Lernfortschritt von Frau Pavithra Sureshkumar}
In unserem Projekt habe ich sehr viel Neues gelernt. Ich habe nicht nur fachliches Wissen erweitert und eingesetzt, sondern auch Teambildung und Teamarbeit ausgeführt. Wir haben trotz dass wir nur drei Personen sind, versucht Scrum auszuführen.\newline
Das ein Team nicht immer einwandfrei läuft ist selbstverständlich. 
Es können immer Probleme bei Kommunikation entstehen und man könnte etwas falsch verstehen oder übermitteln. 
Dafür haben wir eine Retrospektive gemacht. Wir haben unsere Probleme angesprochen und eine Lösung gefunden. trotzdem hat sich das Arbeitsverhalten von Kevin sich nicht geändert.
Auch haben wir gelernt, dass manche nicht die Motivation gehabt haben etwas im Team zu leisten. Ich und Ralf haben beschlossen dann etwas als Team zu unternehmen, um den Teamgeist zu steigern. 
Mit unserem Betreuer haben wir immer versucht jede Woche eine Besprechung zu halten und haben diese auch immer dokumentiert und immer nach der Besprechung zusammengefasst und es dem Meeting Mitglieder geschickt, um den Überblick zu behalten. 
Auch wenn man sich austauscht, sollte aber in jedem Team das selbstständige Arbeiten nicht ausfallen. Mit dem selbstständigen Arbeiten ist gemeint, 
dass man Aufgaben übernimmt und nicht nur die Aufgaben zugeteilt bekommt, das heißt Eigeninitiative sollte bei jedem vorhanden sein. Diese erwies sich schwer, weil es immer verschiedene Arbeitscharaktere gibt. 
Manche können nicht sich organisieren oder Selbstständigkeit vermitteln oder müssen extra Motivation finden. Das alles haben wir im Seminar gelernt und ist für uns eine große Lehre.\newline
Außerdem haben wir die Planung unterschätzt. Wir hätten strenge Regeln setzen müssen, damit die Aufgabenverteilung gerecht verteilt wird. Da Kevin keine Eigeninitiative gezeigt hat mussten ich und Ralf
ihm immer Aufgaben zuweisen. Am Ende hat er diese nicht mal ausgeführt und sich erst später damit beschäftigt und sich nicht mal damit auseinandergestzt wie man es hätte machen sollen.
Ich und Ralf in der Gruppe mehr machen müssen und haben auf Hilfe von Kevin gewartet, die nie ankam. Am Anfang wurde die Planung so ausgeführt, dass sich jeder mit seinem Thema beschäftigt und schon was macht. 
Leider hat Kevin diese Aufgabe nicht ernst genommen und es kam dazu, dass sich ein paar Gruppenmitglieder mit sehr viel Arbeit überbelastet haben. Am Ende haben wir noch versucht die Aufgabenverteilung gerechter zu machen. 
Trotz das ich und Ralf sich mehr Mühe gegeben haben und schon mittlerweile auch mental nicht instande waren auf Kevin aufzupassen, hat er sich nicht einmal nach den Ferien bei uns gemeldet.
Wir sind nicht seine Aufsichtsperson und haben uns auch beschlossen dann auf unsere Aufgaben zu kümmern. Es ist nicht selbstverständlich für mich einer Person hinterher zu laufen,
wenn wir das gleiche Arbeitspensum abliefern müssen. Ich habe dabei nur gelernt, dass ich nächstes Mal mich nicht mit so einer Person annährend beschäftigen möchte und wenn dann es schon früher
dem Betreuer vermitteln würde, auch wenn es sehr offensichtlich war, dass diese Person sich nicht viel Mühe im Projekt gegeben hat.
