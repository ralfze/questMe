\section{Reflektion Lernfortschritt}
Hier wird der Lernforschritt, den wir als Gruppe gesammelt haben beschrieben.

\subsection{Reflektion Lernfortschritt von Frau Pavithra Sureshkumar}
Hier wird der Lernfortschritt von Frau Pavithra Sureshkumar zusammengefasst. Dabei werden nicht nur die teschnischen Lernfortschritte sondern auch die Teamarbeit als Reflektion erläutert.\newline

\noindent In unserem Projekt habe ich sehr viel Neues gelernt. Meine Aufgabe war es im Team das Fronted mit Angular zu programmieren und es aufzustelllen.
Da ich kaum wissen über Typescript oder Angular hatte, musste ich mich erst mal damit beschäftigen, wie ich diese einsetzten kann. Dafür habe ich mich mit der 
Angular docs Seite beschäftigt \href{https://angular.io/docs}{https://angular.io/docs}, auch noch weitere Tutorials angeschaut und natürlich auch bei Problemen in Stackoverflow nachgeschaut. Als Richtline für die Komponenten habe ich Angular Material benutzt, 
abgeändert und versucht diese in das Interface einzusetzen \href{https://material.angular.io/}{https://material.angular.io/}. 
Ich habe gleich in der ersten Woche versucht das Chatinterface während dem Lernen zu programmieren.
Ich habe im Projekt sehr viel mit Ralf gearbeitet, weil ich mit ihm das Chat Interface was ich programmiert habe mit dem backend verbinden musste. Ich habe auch das Angular Routing angewandt um durch die ganzen
Seiten zu navigieren. Nachdem der Chat mit dem backend verbunden wurde habe ich mich mit dem Admin Interface befasst. Dort musste ich das Routing einsetzen und die ganzen Seiten verbinden.
Ich habe mich mit Kevin ausgetauscht, wie es für ihn einfacher wäre auf die Routen zuzugreifen und habe diese auch so umgesetzt, wie er es haben wollte.
Nachdem ich die Komponenten mit Angular im Admin Interface programmiert habe, konnte ich mein Wissen mit Ralf teilen und das backend mit dem frontend der Admin Interface Seiten verbinden, um den Seiten eine Funktion zu erschaffen.
Dabei haben wir beide gelernt, dass es viel einfach gewesen wäre ein Datenmodell von vorne herein zu erschaffen. Ich und Ralf haben es aber trotzdem geschafft das Chat und das Admin Interface so weit wie möglich zu programmieren.
Ich habe mich kurz mit dem CI/CD beschäftigt konnte diese aber nicht durchführen, weil ich andere Aufgaben erledigen musste und kaum Zeit gefunden habe.
Trotzdem habe ich mit Ralf gelernt, wie man eine branch im Gitlab cleaned und wie man mit den erstellten Issues arbeitet.
Alles was ich gemacht und gerlernt habe, habe ich in Gitlab in den Issues ausführlich dokumentiert. 
Ich habe mich nicht nur mit dem programmieren beschäftigt, sondern habe auch Richtlinien für das dokumentieren erstellt und das meinen Teammitgliedern weitergegeben. In der Dokumentation habe ich während ich programmiert habe gearbeitet.
Bei der Zwischenpräsentation und bei der Endpräsentation habe ich den ersten Schritt gewagt für die Gruppe die Präsentationsfolien zu erstellen und die allgemeine Gliederung zu erstellen.
Die Enddokumentation habe ich auch selber angefangen und habe dann mich mit Ralf ausgetauscht. Da Kevin sich nach den Weihnachtsferien nicht gemeldet hat. Habe ich seine Aufgaben ausgelassen und auf eine Ergänzung gewartet.\newline

\noindent Ich habe nicht nur fachliches Wissen erweitert und eingesetzt, sondern auch Teambildung und Teamarbeit ausgeführt. Wir haben trotz dass wir nur drei Personen sind, versucht Scrum auszuführen.
Das ein Team nicht immer einwandfrei läuft ist selbstverständlich. 
Es können immer Probleme bei Kommunikation entstehen und man könnte etwas falsch verstehen oder übermitteln. 
Dafür haben wir eine Retrospektive gemacht. Wir haben unsere Probleme angesprochen und eine Lösung gefunden. Trotzdem hat sich das Arbeitsverhalten von Kevin nicht geändert.
Auch haben wir gelernt, dass manche nicht die Motivation gehabt haben etwas im Team zu leisten. Ich und Ralf haben beschlossen dann etwas als Team zu unternehmen, um den Teamgeist zu steigern. 
Mit unserem Betreuer haben wir immer versucht jede Woche eine Besprechung zu halten und haben diese auch immer dokumentiert und immer nach der Besprechung zusammengefasst und es dem Meeting Mitglieder geschickt, um den Überblick zu behalten. 
Auch wenn man sich austauscht, sollte aber in jedem Team das selbstständige Arbeiten nicht ausfallen. Mit dem selbstständigen Arbeiten ist gemeint, 
dass man Aufgaben übernimmt und nicht nur die Aufgaben zugeteilt bekommt, das heißt Eigeninitiative sollte bei jedem vorhanden sein. Diese erwies sich schwer, weil es immer verschiedene Arbeitscharaktere gibt. 
Manche können nicht sich organisieren oder Selbstständigkeit vermitteln oder müssen extra Motivation finden. Das alles haben wir im Seminar gelernt und ist für uns eine große Lehre.\newline
Außerdem haben wir die Planung unterschätzt. Wir hätten strenge Regeln setzen müssen, damit die Aufgabenverteilung gerecht verteilt wird. Da Kevin keine Eigeninitiative gezeigt hat mussten ich und Ralf
ihm immer Aufgaben zuweisen. Am Ende hat er diese nicht mal ausgeführt und sich erst später damit beschäftigt und sich nicht mal damit auseinandergestzt, wie man es hätte machen sollen.
Ich und Ralf in der Gruppe mehr machen müssen und haben auf Hilfe von Kevin gewartet, die nie ankam. Am Anfang wurde die Planung so ausgeführt, dass sich jeder mit seinem Thema beschäftigt und schon was macht. 
Leider hat Kevin diese Aufgabe nicht ernst genommen oder halbherzig ausgeführt und es kam dazu, dass sich ein paar Gruppenmitglieder mit sehr viel Arbeit überbelastet haben. Am Ende haben wir noch versucht die Aufgabenverteilung gerechter zu machen. 
Trotz das ich und Ralf sich mehr Mühe gegeben haben und schon mittlerweile auch mental nicht instande waren auf Kevin aufzupassen, hat er sich nicht einmal nach den Ferien bei uns gemeldet.
Wir sind nicht seine Aufsichtsperson und haben uns auch beschlossen dann auf unsere Aufgaben zu kümmern. Es ist nicht selbstverständlich für mich einer Person hinterher zu laufen,
wenn wir das gleiche Arbeitspensum abliefern müssen. Ich habe dabei nur gelernt, dass ich mich nächstes Mal nicht mit so einer Person annährend beschäftigen möchte und wenn, dann es schon früher
dem Betreuer vermitteln würde, auch wenn es sehr offensichtlich war, dass diese Person sich nicht viel Mühe im Projekt gegeben hat. 
Ich sehe dieses Projekt als eine große Lehre und werde in den nächsten Projekten, die vor mir stehen meine Erfahrungen die aus diesem Projekt hervorgegangen sind Einsetzten.
