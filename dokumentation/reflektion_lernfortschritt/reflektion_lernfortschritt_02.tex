\newpage
\subsection{Reflektion Lernfortschritt von Herr Ralf Zeller}

\noindent Am Anfang des Projektes ergaben sich Aufgabenbereiche Frontend- , Backend- Entwicklung, Präsentation und Dokumentation.
Bei der Aufgabenverteilung haben Pavithra und ich uns aktiv daran beteiligt Aufgaben anzunehmen und diese zeitlich möglichst früh umzusetzen.
Damit Kevin ebenfalls Aufgaben übernehmen konnte, mussten wir als Team direkt vorschlagen welche Aufgaben er tun sollte,
da er sich nicht in das Team eingebracht hat und auch keinen Vorschlag selbständig erbracht hat.
Dadurch ergab sich eine unausgeglichene Arbeitslastverteilung für die gesamte Gruppe,
da wir nicht einschätzen konnten wie viel zeitlichen Aufwand er mit den zugeteilten Aufgaben noch benötigt oder
er neue Aufgaben annehmen konnte.
Pavithra hat als Hauptaufgabe Angular bearbeitet und die benötigten Datenmodelle für das MVC Pattern von Angular erstellt.
Kevins Hauptaufgabe ist die Integration von Keycloak und Basis für die Technologien zu schaffen und diese zu erweitern.
Meine Hauptaufgabe ist das NLP.js Framework/Bibliothek in unser Projekt zu integrieren,
eine Verbindung aller Technologien zu erstellen und eine Alternativen für NLP.js zu recherchieren,
falls es sich nicht umsetzen lässt.

\noindent In den ersten Wochen des Projektes hat Pavithra Recherchen für ihren Aufgabenbereich Angular erledigt und erste Prototypen für das Angular UI erstellt.
Kevin hat in dieser Zeit seine Recherche für Keycloak weitergeführt.
Mein Aufgabenbereich war Aufzuklären ob das Framework Nlp.js der AXA/Group in unser Projekt integrierbar ist und zu testen,
ob die vorgeschlagenen Technologien mit diesem Framework funktionieren.
Dabei habe ich aus den Präsentationsfolien von Herr Renz,
beide vorgeschlagenen Chatbot Guides (Paresh Joshi Building a simple Chatbot Application und Jesús Seijas Getting started with NLP.js)
bearbeitet, den Quellcode zum Laufen gebracht und beide Chatbots bei unseren regelmäßigen Gruppen Meetings vorgestellt.
Damit die gesamte Gruppe nachvollziehen konnte in welcher Form wir den Chatbot in unserem Projekt realisieren können.
Nach dieser Zeit musste die Zwischenpräsentation vorbereitet werden.
Die ersten Versionen wurden von Pavithra erstellt, mögliche Gliederungsoptionen ausprobiert und erfolgreich übernommen.
Während dieser Zeit war Kevin weiterhin beschäftigt mit den Recherchen für Keycloak.
Meine Aufgabe war es während dieser Zeit in Figma die geeigneten Darstellungen für die Zwischenpräsentation zu erstellen.
Sowie die erstellten UI Prototypen von Pavithra in Figma zu erweitern und auszubessern.
Damit wir gemeinsam die Präsentation Üben konnten mussten wir Kevin direkt auffordern sich Zeit zu nehmen und zur Hochschule zu kommen,
um die Präsentation zu üben und zu testen.
Nachdem Meilenstein der Präsentation war Pavithra damit beschäftigt Gitlab für den präsentierten Projektablauf in Gitlab anzupassen und zu organisieren.
Darunter fielen das Erstellen der Meilensteine, erste Issues die erledigt werden sollten und die Priorisierungsstruktur für die Aufgaben die übernommen werden, an.
Kevin war während dieser Zeit weiterhin dabei zu recherchieren wie Keycloak in unserem Projekt integriert werden kann.
Währenddessen habe ich die Struktur für unsere Dokumentation in Latex erstellt und alle Gruppen Mitglieder informiert wie wir Latex zum Erstellen unserer Dokumentation verwenden können.
Anschließend haben wir in der Gruppe vereinbart wer welche Aufgabenbereiche in unsere Dokumentation überträgt.

\noindent Indem die meisten Technologien, die wir verwendet hatten sehr neu für uns waren ergab sich für jeden eigener großer Lernaufwand.
Nicht jeder in der Gruppe zeigte die gleiche Motivation und Interesse für das Projekt, um die Wissenslücken nachzuholen und die Technologien für unser Projekt einsetzbar zu machen.
Somit mussten Pavithra und ich die meisten Aufgaben bearbeiten und immer wieder dazu Appellieren, die angenommen Aufgaben möglichst frühzeitig zu erledigen und Ergebnisse zu liefern.
Damit man wieder Zeit für neue Aufgaben hat, die essentiell für das Projekt sind und diese bearbeiten zu können.
Zusätzlich erschwerte Kevin die Gruppenarbeit durch seine fehlende Eigeninitiative im Team, die dazu führte das Pavithra und ich Kevin Aufgaben direkt zuteilen mussten.
Bei uns entstand der Eindruck, dass jeder Mehraufwand zu viel war und nur das nötigste von ihm bearbeitet werden soll.
Als Beispiel wurden verfasste Texte einfach kopiert und um minimale Worte verändert wie es bei den Use Cases der Fall ist.
Anstatt vordefnierte Texte zu verwenden und diese umzuformen.
In unserer Anwendung wurde das UI von Pavithra in Angular erstellt und in Zusammenarbeit mit mir mit dem Backend verbunden.
Zusätzlich haben wir gemeinsam das UI soweit es möglich war verfeinert.
Für die Verbindung der Datenbank wurde von mir ein REST Api mit den essentiellen Methoden für das Projekt erstellt.
Kevin hat das REST API, um geforderte Funktionen ergänzt.
Weitere Funktionen des REST Api für die Seiten General und Einstellungen wurden von mir im REST Interface integriert und als Datenmodell in mongoDb übernommen.
Von Kevin wurde Keycloak in unsere Anwendung integriert.
Darunter fiel die Einbindung in den Nodejs Server und in das Angular UI.
Von mir wurde eine Integration des Frameworks/Bibliothek NLP.js in unsere Anwendung umgesetzt.
Sowie die benötigten Verbindungen aller Technologien mit Socket.io und REST.

\noindent Unser Projekt besteht aus vier Teilen Nodewebapp, Angular-Frontend, mongoDb und Keycloak.
Die Dockerfiles für jeden Docker Container wurde von mir erstellt.
Für Angular habe ich mir Hilfe bei Pavithra geholt, um das Dockerfile zu erstellen.
Und für das Erstellen des Dockerfile für Keycloak habe ich mir die benötigten Informationen von Kevin erfragt.
Anschließend wurde von mir eine Docker Compose eingerichtet, dass das Erstellen unserer Anwendung für jedes Teammitglied erleichtert hat.

\noindent In unseren wöchentlichen Meetings haben Pavithra und ich unsere neuen Erkenntnisse und Erfahrungen vorgestellt.
Damit jeder über den Fortschritt der bearbeiteten Aufgaben informiert ist.
Zusätzlich haben Pavithra und ich alle Erkenntnisse, Probleme in Gitlab in Issues dokumentiert.
Von Kevin wurden sehr zaghaft Erkenntnisse und Ergebnisse in den Meetings vorgestellt .
Wodurch auch einige Wochen keine neue Informationen zum Stand seiner bearbeiteten Aufgaben im Team vorgestellt wurden.
Aufgrund der fehlenden Bereitschaft im Meeting das Team über seinen derzeitigen Wissensstand aufzuklären,
haben wir als Gruppe an Kevin appelliert seine Bearbeitung der Aufgaben in Gitlab wie wir ebenfalls zu dokumentieren.

\noindent Aufgrund der mangelnden Bereitschaft von Kevin sich im Team mehr einzubringen, haben Pavithra und ich beschlossen eine Retrospektive durchzuführen.
In dieser Retrospektive haben alle Gruppenmitglieder sich gemeinsam ausgesprochen und versucht eine Lösung für die Probleme des Teams zu finden.
In den ersten zwei Wochen hat sich die Retrospektive sehr positiv auf die Bearbeitung des Projekts ausgewirkt.
Wodurch ich die Verbindung des Angular UIs mit dem Backend in Zusammenarbeit mit Pavithra besser bearbeiten konnte.
Kevin hat dabei das vorliegende REST Api erweitert. Nach den zwei Wochen ist die Bearbeitung wieder größtenteils auf Pavithra und mich zurückgefallen.
Dadurch, dass es sehr viele offene Fragen gab die bearbeitet werden mussten,
haben wir kaum Zeit gefunden als Team zu besprechen wie wir das gemeinsam schaffen möchten und waren dazu gezwungen so viele Aufgaben wie möglich zu übernehmen.
Kevin hat sich nicht informiert welche Aufgaben noch offen stehen und auch nicht danach gefragt, ob wir die Aufgaben besser aufteilen können.
Er hat uns zugesichert er würde ein Sicherheitskapitel für den Chatbot erstellen und uns informieren sobald er es erledigt hat, um weitere Aufgaben bearbeiten zu können.

\noindent Gegen Ende des Projektes wurden viele Teile der Dokumentation und Ausbesserung der Anwendung von Pavithra und mir kontinuierlich bearbeitet.
Damit die notwendigen Vorgaben weitgehend erfüllt werden konnten.
Wahrscheinlich wurde von Kevin es so verstanden, dass er nur für Keycloak zuständig ist. 
Dadurch keine anderen Aufgaben des gesamten Projektes übernehmen und nur Aufgaben übernehmen muss die ihm zugeteilt werden.
Trotzdem wurde innerhalb der Gruppe immer wieder dazu appeliert, dass sich jeder der Aufgaben erledigt hat wieder mehr an der Bearbeitung des gesamten Projektes zu beteiligen.

\noindent Bei diesem Projekt habe ich gelernt wie man mit den Technologien Angular, Node.js, Socketio, Keycloak und Nlp.js einen sehr umfangreichen Chatbot entwickelt.
Dabei habe ich gelernt, wie man dynamische Inhalte in Angular gestaltet und welche Informationen vom Backend als Service bereitgestellt werden sollten.
Einen REST Service entwickelt und welche Aufgaben dieser zu erfüllen sollte. Welche Möglichkeiten Keycloak bietet, um Addressen der Anwendung abzusichern.
Mit Socketio habe ich gelernt eine bidirektionale Verbindung zwischen dem Angular Server und dem Node.js Server herzustellen.
Meine Hauptaufgabe war zu Erlernen wie man NLP.js der AXA/Group für unser Projekt verwendet und das Framework vorteilhaft in unserer Anwendung integriert.
Wegen diesem Framework/Bibliothek musste ich sehr viel Neues darüber lernen, wie das Framework funktioniert und wie man es am Besten integriert.
Dadurch habe ich sehr viel Zeit in Recherchen und Tests für die Integration des Frameworks für das Projekt gesteckt.
Außerdem habe ich sehr viel Wissen und Informationen darüber gewonnen wie Texte verarbeitet werden.
Dabei habe ich die Oberfläche von Natural Language Processing angschnitten, indem ich mir die für das Projekt notwendigen Kenntnisse angeeignet habe.
Beim Mitorganisieren und Durchführen des Projektes habe ich gelernt wie man mit Gitlab ein Projekt sehr gut plant und durchführt.
Sowie die anderen Teammitglieder ausreichend informiert über die Aufgaben, die man erledigt hat. Wie man Issues mit Commits sehr gut verbindet,
um Erklärungen für bestimmte Aufgaben oder Änderungen nachweisen zu können.
Außerdem wie man in Gitlab die Struktur von Branches und Merges sauber und nachvollziehbar ordnet und erstellt.
Für die Dokumentation haben wir Latex als Grundlage verwendet,
wodurch ich sehr viele Erfahrungen im Umgang mit Latex gewonnen habe.
Die gewonnene Erfahrung erleichtert mir das Erstellen eines Latex Dokuments und der späteren Erstellung der Bachelorarbeit.
In der Gruppenarbeit habe ich gelernt, dass nicht jeder der beteiligt ist am Thema des Projektes sich gleichermaßen begeistern kann und entsprechend Zeit dafür aufwänden möchte,
um ein sehr gutes Ergebnis zu erbringen. Dass man bei Personen, die sich aus der Bearbeitung von Aufgaben heraushalten möchten,
feste Termine setzen und Ergebnisse sehr frühzeitig fordern muss.
Aufgaben die für das Projekt anfallen mit allen Teammitgliedern besser besprechen sollte und auch Aufgaben an jemanden abgibt,
dem man nicht zutraut die Aufgaben rechtzeitig bearbeiten zu können.

\noindent In allem hat mir das Bearbeiten des Projektes sehr viel Spaß gemacht, da ich sehr viel Lernen konnte.
Ich hätte mir gewünscht, dass die Aufgaben des Projektes auf alle Gruppenmitglieder besser verteilt gewesen wären.
Und dadurch mehr produktive Gespräche zum Erweitern unserer Anwendung sich ergeben hätten. 
Als wichtige Erfahrung werde ich mir Merken, sollte ich wieder mit einem Teammitglied arbeiten, 
dass sich kaum an der Teamarbeit beteiligt werde ich es direkt mit dem Betreuer besprechen.
Damit ich selber nicht noch mehr Aufgaben erledigen muss und sehr viel Zeit aufwände Defizite auszugleichen.