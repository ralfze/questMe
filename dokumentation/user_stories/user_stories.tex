\section{User Stories}
In diesem Kapitel haben wir unsere User Stories gesammelt.

\subsection{Struktur der User Stories}
Als \textbf{<Akteur>} möchte ich \textbf{<Funktion>}, um \textbf{<Nutzen>} zu erreichen.
\subsection{User Stories Version 1}
Hier listen wir unsere ersten Ideen auf, die wir erfüllen möchten.
\\

\textbf{Chatfenster}
\begin{enumerate}[leftmargin=*,labelindent=40pt,label=u\arabic*.]
    \setcounter{enumi}{10000}
    \item Als Nutzer möchte ich eine Nachricht abschicken können, um mit dem Chatbot zu interagieren.
    \item Als Nutzer möchte ich eine Frage stellen können, um eine Antwort zu erhalten.
    \item Als Nutzer möchte ich eine Antwort dargestellt bekommen, um die Antwort lesen zu können.
    \item Als Nutzer möchte ich meine Nachrichten, von denen des Bots unterscheiden können, um zu erkennen, welche Nachricht die Antwort ist.
    \item Als Nutzer möchte ich darauf hingewiesen werden, wo ich zu Schreiben habe, um eine Nachricht verfassen zu können.
\end{enumerate}


\begin{enumerate}[leftmargin=*,labelindent=40pt,label=u\arabic*.]
    \item[\textbf{Hinweis:}] Ein Teil der Professoren übernimmt administrative Tätigkeiten des Chatbots. Dadurch ist mit dem erwähnten Administrator immer ein administrativ tätiger Professor gemeint.
\end{enumerate}
\newpage
\textbf{Admin Interface: Allgemein}
\begin{enumerate}[leftmargin=*,labelindent=40pt,label=u\arabic*.]
    \setcounter{enumi}{20000}
    \item Als Admin möchte ich eine Möglichkeit haben, um zwischen den Seiten des Admin-Interfaces wechseln zu können.
    \item Als Admin möchte ich verschiedene optische Konfigurationen zur Auswahl haben, um mein ChatBot zu individualisieren.
    \item Als Admin möchte ich erkennen können, welche Option ich ausgewählt habe, um zu wissen, was aktuell ausgewählt ist.
    \item Als Admin möchte ich den Namen des ChatBots ändern, um meinen ChatBot zu individualisieren.
    \item Als Admin möchte ich ein Eingabefeld erkennen können, um zu wissen, wo ich etwas eingeben kann.
    \item Als Admin möchte ich die ausgewählte Seite des Admin-Interfaces in einer anderen Farbe sehen, um zu erkennen auf welcher Seite ich bin.
\end{enumerate}

\textbf{Admin Interface: Korpus}
\begin{enumerate}[leftmargin=*,labelindent=40pt,label=u\arabic*.]
    \setcounter{enumi}{30000}
    \item Als Administrator möchte ich eine Liste mit allen Fragen und Antworten, um einen Überblick über den Korpus zu haben.
    \item Als Administrator möchte ich einen neuen Eintrag hinzufügen, um neue Fragen und Antworten hinzuzufügen zu können.
    \item Als Administrator möchte ich eine Möglichkeit zum Hinzufügen von Fragen, um neue Fragen hinzuzufügen zu können.
    \item Als Administrator möchte ich eine Möglichkeit zum Hinzufügen von Antworten, um neue Antworten hinzuzufügen zu können.
    \item Als Administrator möchte ich eine Möglichkeit zum Entfernen von Fragen, um Fragen entfernen zu können.
    \item Als Administrator möchte ich eine Möglichkeit zum Entfernen von Antworten, um Antworten entfernen zu können.
    \item Als Administrator möchte ich eine Möglichkeit einen Eintrag „Fragen und Antworten“ bearbeiten zu können, um den Eintrag zu ändern.
\end{enumerate}
\newpage
\textbf{Node.js Allgemein}
\begin{enumerate}[leftmargin=*,labelindent=40pt,label=u\arabic*.]
    \setcounter{enumi}{40000}
    \item Als Nutzer möchte ich eine bidirektionale Kommunikation zwischen dem Client und Server, um direkt mit dem Bot kommunizieren zu können.
    \item Als Nutzer möchte ich, dass der ChatBot meinen Kontext versteht, um mit dem Bot nach Kontext zu chatten.
    \item Als Administrator möchte ich die Möglichkeit den Korpus des ChatBots persistent zu speichern, um auf den Korpus zuzugreifen zu können.
    \item Als Nutzer möchte ich, dass der ChatBot über eine Webadresse erreichbar ist, um mit dem ChatBot online zu kommunizieren.
\end{enumerate}

\textbf{KeyCloak}
\begin{enumerate}[leftmargin=*,labelindent=40pt,label=u\arabic*.]
    \setcounter{enumi}{50000}
    \item Als Admin möchte ich mich in KeyCloak einloggen können, um es zu verwalten.
    \item Als Admin möchte ich mich in das Admin-Interface einloggen können, um die Einstellungen des Chatbots zu verwalten.
    \item Als Hochschulangehöriger möchte ich mich mit dem Shibboleth SSO der Hochschule einloggen, um relevante Daten mitzuteilen.
    \item Als Admin möchte ich schnellen Zugriff auf das KeyCloak-Webinterface über das Admin-Interface, um Zeit zu sparen.
    \item Als Admin möchte ich neue Nutzergruppen erstellen, um zielgerichteter Fragen beantworten zu können.
    \item Als Admin möchte ich einer Nutzergruppe einen neuen Fragensatz zuweisen, um die möglichen Fragen für diese Gruppe zu erweitern.
    \item Als Admin möchte ich einen, zu einer Nutzergruppe zugewiesenen, Fragensatz entfernen, um möglichen Fragen für diese Gruppe einzuschränken.
    \item Als Admin möchte ich Nutzer verwalten, um bei Bedarf Änderungen vorzunehmen.
    \item Als Admin möchte ich die Login-Seite anpassen, um sie nach meinen Vorstellungen zu ändern.
\end{enumerate}



