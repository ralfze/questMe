\section{Usability Test}
Hier werden wir unser Vorgehen des Usability Tests beschreiben und die ausführliche
Durchführung des Tests.

\subsection{Kriterien für Usability}
Unsere Kriterien für Usability, die uns wichtig sind und für und unser User Interface
essentiell sind, werden wir hier kurz erläutern.

\subsubsection{Responsive Webdesign}

\noindent Wir möchten unseren ChatBot auf allen Geräten ohne Probleme ausführen lassen.
Das heißt, dass wir auch auf mobilen Geräten unsere Software im Browser laufen lassen wollen.
Die mobile Seite sollte dann auch auf die wichtigsten Elemente beschränkt werden, damit
der Benutzer es leichter hat die Icons treffsicher mit ihren Fingern zu benutzen.

\subsubsection{Gute Lesbarkeit}

\noindent Unsere Anwendung sollte leicht zu lesen sein, weil das Lesen auf dem Bildschirm
grundsätzlich schwieriger ist. Deswegen sollten wir auf jeden Fall
auf Textgröße und Kontrastreiche Farben achten. Die Textgröße sollte mindestens
12pt haben. Auch sollten wir lange Textblöcke und
Schachtelsätze vermeiden.

\subsubsection{Gute Navigation}

\noindent Eine übersichtliche und eine verständliche Navigation ist das wichtigste bei einer Webseite.
Eine gute Navigation verhindert Verwirrung und unterstützt den Benutzer zu seinem Ziel zu gelangen.
Bei der Navigation sollte man darauf achten, dass alle Verlinkungen funktionieren.

\subsubsection{Schnelle Ladezeiten}

\noindent Die Webseite sollte ihre Inhalte schnell laden und keine großen Verzögerungen
aufzeigen. Sie sollte bei ungefähr drei Sekunden liegen, um keine User zu verlieren.

\subsubsection{Interessantes Design}

\noindent Eine Konsistente Einhaltung von bestimmten Farben ist sehr wichtig. Der erst Eindruck von
einer Webseite zeigt schon, ob die User die Webseite benutzen möchten oder nicht.
So können Benutzer durch Bilder mit schlechter Qualität oder zu grellen Farben abgeschreckt werden.
Pop-Ups sollten in Grenzen gehalten werden oder vermieden werden.

\subsection{Unser Usability Test}
Hier beschreiben wir welche Methode wir zum Testen unserer Usability Kriterien benutzen
möchten.

\subsubsection{Remote User Testing}

\noindent Zuerst planen wir was wir testen möchten und dann, wie wir testen möchten. In unserem Fall haben wir uns
für remote testing entschieden. Als nächstes überlegen wir uns besondere Tasks, die der User absolvieren muss, um
zu erkennen, ob unsere Kriterien eingehalten werden und was wir Verbessern sollten.
Die Szenarien sollten so einfach und realistisch sein, dass der Benutzer es leicht durchführen kann.
Dann müssen wir auch Tester finden, welche in unserer Zielgruppe passen.

\subsubsection{Durchführung der Remote Usability Testsitzung}

Der erste Schritt beinhaltet den Usability-Testplan. Dieser beinhaltet den Zweck des Tests, 
die Kosten und Zeiteinschätzung zur Durchführung des Tests, das Testskript mit den Usability-Testaufgaben und die Rekrutierung der Testteilnehmer. 
\\

\noindent Bei der Rekrutierung werden die User so ausgesucht, dass am besten alle Zielgruppen gedeckt sind. Jeder Testteilnehmer/in wird einzeln durch das Usability Test durchgeführt. 
Geplant sind insgesamt vier bis sechs Teilnehmer. Nach dem Aussuchen der Testteilnehmer wird am Tag des Tests die Einverständniserklärung für die Teilnahme und Datenschutz von den Usern unterschrieben. 
Die Einverständniserklärung sollte also schon vorbereitet sein. Auch die Tasks werden von vorneherein mitbestimmt. Nach dem Unterschreiben werden die Testteilnehmer mittels eines Briefings benachrichtigt, wie der Test abläuft und was zu beachten ist. 
Außerdem wird ausdrücklich auch vermittelt, dass es kein Test ist, um ihre Leistungsfähigkeit zu beurteilen, sondern dient nur für die Weiterentwicklung und zur Bewertung unserer Software.
\\

\noindent Als nächstes findet die Pre-Session statt. In dieser Session finden wir heraus, welche Erfahrungen die Testteilnehmer mit dem zu testenden System hat und welche Interessen er vertritt und was er von Chatbots hält. 
Nach der Pre-Session werden die Testteilnehmer gebeten, laut-denkend ihre Testaufgaben durchzuführen. 
Der Moderator sollte still zuhören und beobachten. Ganz wichtig ist es nicht bei Schwierigkeiten einzugreifen, weil es sonst den Test manipuliert.
\\

\noindent Eine Aufgabe könnte sein den Testteilnehmer zu bitten, dass Sie den Chatbot dazu bringen eine Information zu vermitteln, welche die Teilnehmer haben möchten. 
Der Testteilnehmer könnte Fragen stellen, welche der Chatbot kennt oder auch nicht. 
Der Moderator schreibt sich auf, wie der Testteilnehmer auf sein Ergebnis kommt, oder auch gewisse Schwierigkeiten zeigt und bei Aufgaben stockt. 
Es könnte sein, dass man dann den Korpus erweitern oder anpassen müsste. 
Alles wird gründlich dokumentiert, während der Testteilnehmer laut-denkend seine Aufgaben erledigt.
\\

\noindent Bei uns ist der Moderator der Protokollant, da es remote abläuft. Wir überlegen außerdem die Session aufzunehmen. 
\\

\noindent Abschließend, in der Post-Session, werden dann die Teilnehmer auf den Gesamteindruck des zum testenden System befragt. 
Es ist sehr wichtig eine ausführliche Dokumentation zu schreiben, um jeden Eindruck und jedes Vorgehen festzuhalten.
\\

\noindent Nach der Usability-Testsitzung werden die Befunde zusammengeschrieben und die Ergebnisse werden ausgewertet. 
Anschließend wird der Usability-Testbericht geschrieben. 
Der Usability-Testbericht beinhaltet nicht nur die Usability-Probleme, sondern auch die gelungenen Usability-Befunde.
