\section{Reflektion Projektmanagement}
In diesem Abschnitt wird Abweichung der geplanten Termine von den tatsächlichen Termine des Projektes verglichen.
Reflektiert wird, ob die Planung wie gehabt durchgeführt wurde und 
wie die Aufwandschätzung von der Planung und der Realisierung war.


\begin{table}[H]
    \begin{center}
        \begin{tabular}{|p{180px}|c|r|}
            \hline
            \begin{tabular}{l}
                \\
                \large \bfseries Meilensteine \\
                \\
            \end{tabular}
             &
            \begin{tabular}{l}
                \\
                \large \bfseries alter Stand \\
                \\
            \end{tabular}
            &
            \begin{tabular}{l}
                \\
                \large \bfseries neuer Stand \\
                \\
            \end{tabular}
            \\
            \hline
            \begin{tabular}{l}
                \\
                \large Recherche \\
                \\
            \end{tabular}
             &
            \begin{tabular}{l}
                \\
                \large \bfseries 13.10.21 \\
                \\
            \end{tabular}
            &
            \begin{tabular}{l}
                \\
                \large \bfseries \color{red} 11.01.22 \\
                \\
            \end{tabular}
            \\
            \hline
            \begin{tabular}{l}
                \\
                \large Zwischenpräsentation \\
                \\
            \end{tabular}
             &
            \begin{tabular}{l}
                \\
                \large \bfseries 22.10.21 \\
                \\
            \end{tabular}
            &
            \begin{tabular}{l}
                \\
                \large \bfseries \color{green} 22.10.21 \\
                \\
            \end{tabular}
            \\
            \hline
            \begin{tabular}{l}
                \\
                \large Implementation \\
                \\
            \end{tabular}
             &
            \begin{tabular}{l}
                \\
                \large \bfseries 25.10.21 \\
                \\
            \end{tabular}
            &
            \begin{tabular}{l}
                \\
                \large \bfseries \color{red} 12.01.22 \\
                \\
            \end{tabular}
            \\
            \hline
            \begin{tabular}{l}
                \\
                \large MVP und Code Review \\
                \\
            \end{tabular}
             &
            \begin{tabular}{l}
                \\
                \large \bfseries  02.12.21 \\
                \\
            \end{tabular}
            &
            \begin{tabular}{l}
                \\
                \large \bfseries \color{red} 10.12.21 \\
                \\
            \end{tabular}
            \\
            \hline
            \begin{tabular}{l}
                \\
                \large \bfseries \color{blue} Endpräsentation und \\
                \large \bfseries \color{blue} Enddokumentation\\
                \\
            \end{tabular}
             &
            \begin{tabular}{l}
                \large \bfseries 14.01.22 \\
            \end{tabular}
            &
            \begin{tabular}{l}
                \large \bfseries \color{green} 14.01.22 \\
            \end{tabular}
            \\
            \hline
        \end{tabular}
        \caption{Meilenstein Liste: Neuer Stand}
        \label{tab:Meilenstein Liste Neuer Stand}
    \end{center}
\end{table}


   


\subsection{Geplante Meilensteine für Reflektion}
Hier werden die Meilensteine, die man als Ziel gesetzt hat verglichen 
mit dem realen Aufwand.

\subsubsection{Recherche 11.01.22}
Die Recherche lief nach Plan. Dennoch war es nicht möglich bis zum 13.10.21 alles zu erlernen. 
Indem wir sehr viele Bereiche zu erfüllen hatten und nur drei Personen waren,
blieb uns nur übrig, dass wir kontinuierlich Lernen während dem Ablauf des ganzen Projektes. 
So musste nicht nur ein Bereich gelernt werden, sondern wir mussten uns Austauschen, unsere Codes vergleichen und zusammenfügen. 
Dies hat uns noch mehr Zeit gekostet als erwartet und wir sind auf kontinuierliche Recherche umgestiegen.

\subsubsection{Zwischenpräsentation 22.10.21}
Bei der Zwischenpräsentation haben wir uns sehr viel Zeit genommen eine kreative Idee auszudenken, 
was uns von den anderen Gruppen unterscheidet und wie wir unsere Präsentation so einfach wie möglich darstellen können. 
Die Zwischenpräsentation mussten wir auch sehr schnell erledigen und hatten kaum Zeit zu üben, da diese nach einem kurzen Zeitraum stattfand. 
Wir haben uns aber trotzdem in der Hochschule getroffen und haben im Präsentationsraum unsere Präsentation Test geprobt. 
Damit wir vorbereitet unsere Präsentation halten können, unsere Ressourcen kennen (Ton, Beamer,...) und das Reden mit der Mundschutzmaske ausprobieren.

\subsubsection{Implementation 12.01.22}
Den größten Teil der Implementation hatten wir bis zum 12.01.22 erledigt. 
Danach fehlte noch der Feinschliff für die Anwendung.
Das wichtigste war für uns hierbei, dass das Produkt soweit wie möglich fertig ist, damit vorgestellt werden kann.
Die Implementation lief sehr schwergängig, da wir sehr viel Neues gelerntes umsetzen mussten und gleichzeitig wieder Neues erlernen mussten.
Hinzu kam es noch, dass sich Probleme bei der Implementierung ergaben, aber diese wurde in unserer Gruppe durch Pair Programming gelöst.

\subsubsection{MVP und Codereview 10.12.21}
Das MVP oder auch technischer Durchstich genannt haben wir sehr gut hinbekommen, jedoch mussten wir den Termin um eine Woche verschieben, 
um Fehler auszuschließen und neue Features, die wir integriert haben zu testen.
Das wichtigste hierbei war das Codereview. Beim Codereview mussten wir unsere Codes vorstellen, damit wir auch zeigen konnten wer welche Aufgaben gelöst hat. 
Wir wollten unser MVP mehr ausarbeiten, aber hatten dafür zeitliche Probleme, Wissenslücken zu füllen und 
Probleme den Code aller Teammitglieder funktional zu kombinieren, damit die Anwendung auch funktioniert. 

\subsubsection{Endpräsentation und Enddokumentation 14.01.22}
Bei der Endpräsentation haben wir uns vorgenommen diese früher Anzufertigen, aber wir konnten es aus zeitlichen Gründen nicht eine Woche früher vorbereiten.
Es standen noch viele Ausbesserungen des Chatbots aus. 
Die Präsentation kann nur vorbereitet werden, wenn der ChatBot fertig und vorführbar ist. 
Geplant war die Implementation bis zum 12.01.22 fertig zu haben. 
Damit wir noch Zeit bis zu der Präsentation haben, um diese zu üben,
und so gut wie möglich vorbereitet in die Präsentation gehen können.
Die Enddokumentation wurde von Anfang an verfasst und wird dann mit der Endpräsentation abgegeben.