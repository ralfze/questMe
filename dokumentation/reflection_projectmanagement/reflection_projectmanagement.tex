\section{Reflektion Projektmanagement}
In diesem Abschnitt wird die Planung, die man vorgenommen hat
(in Meilensteine) reflektiert. 
Reflektiert wird, ob die Planung wie gehabt durchgeführt wurde und 
wie die Aufwandschätzung vom Geplanten und der Realisierung war.

\begin{table}[H]
    \begin{center}
        \begin{tabular}{|p{180px}|c|}
            \hline
            \begin{tabular}{l}
                \\
                \large Recherche \\
                \\
            \end{tabular}
             &
            \begin{tabular}{l}
                \\
                \large \bfseries 13.10.21 \\
                \\
            \end{tabular}
            \\
            \hline
            \begin{tabular}{l}
                \\
                \large Zwischenpräsentation \\
                \\
            \end{tabular}
             &
            \begin{tabular}{l}
                \\
                \large \bfseries 22.10.21 \\
                \\
            \end{tabular}
            \\
            \hline
            \begin{tabular}{l}
                \\
                \large Implementation \\
                \\
            \end{tabular}
             &
            \begin{tabular}{l}
                \\
                \large \bfseries 25.10.21 \\
                \\
            \end{tabular}
            \\
            \hline
            \begin{tabular}{l}
                \\
                \large MVP \\
                \\
            \end{tabular}
             &
            \begin{tabular}{l}
                \\
                \large \bfseries  02.12.21 \\
                \\
            \end{tabular}
            \\
            \hline
            \begin{tabular}{l}
                \\
                \large \bfseries Endpräsentation und \\
                \large \bfseries Enddokumentation\\
                \\
            \end{tabular}
             &
            \begin{tabular}{l}
                \large \bfseries 14.01.22 \\
            \end{tabular}
            \\
            \hline
        \end{tabular}
        \caption{Meilenstein Liste}
        \label{tab:Meilenstein Liste}
    \end{center}
\end{table}

\subsection{Geplante Meilensteine für Reflektion}
Hier werden die Meilensteine, die man als Ziel gesetzt hat verglichen 
mit dem realen Aufwand.

\subsubsection{Recherche 13.10.21}
Die Recherche lief nach Plan. Was aber nicht ging, alles bis 13.10.21 zu lernen, 
da wir sehr viel zu Erfüllen hatten und nur drei Personen waren, mussten wir kontinuierlich Lernen. 
So musste nicht nur ein Bereich gelernt werden, sondern wir mussten uns Austauschen und unsere Codes vergleichen und zusammenfügen. 
Dies hat uns dann noch mehr Zeit gekostet als erwartet und wir sind auf kontinuierliche Recherche umgestiegen.

\subsubsection{Zwischenpräsentation 22.10.21}
Bei der Zwischenpräsentation haben wir uns sehr viel Zeit genommen uns eine kreative Idee auszudenken, 
was uns von den anderen Gruppen unterscheidet und wie wir unsere Präsentation so einfach wie möglich darstellen. 
Die Zwischenpräsentation mussten wir auch sehr schnell erledigen und hatten kaum Zeit zu üben, da diese nach einem kurzen Zeitraum stattfand. 
Wir haben uns aber trotzdem in der Hochschule getroffen und haben im Präsentationsraum unsere Präsentation Test geprobt, 
um uns vorzubereiten und unsere Ressourcen auszutesten, wie Ton, Laptop verbinden und das Reden mit der Mundschutzmaske nicht zu vergessen.

\subsubsection{Implementation 25.10.21}
Die Implementation dauert noch bis 11.01.22 an und ist gerade dabei noch ein Feinschlief durchzulaufen. 
Das wichtigste für uns ist hierbei das Produkt so weit wie möglich präsentabel zu programmieren. 
Die Implementation lief sehr schwergängig, da wir auch sehr viel zu tun hatten mit nur drei Gruppenmitglieder. 
Hinzu kam es noch, dass sich Probleme beim Implementieren ergeben haben, aber diese von uns wieder durch Pair Programming gelöst wurde. 

\subsubsection{MVP und Codereview 02.12.21}
Das MVP oder auch technischer Durchstich genannt haben wir sehr gut hinbekommen. 
Das wichtigste hierbei war das Codereview. Beim Codereview mussten wir unsere Codes vorstellen, damit wir auch zeigen können wer welche Aufgaben gelöst hat. 
Wir wollten unser MVP mehr ausreifen, aber hatten wieder das zeitliche Problem, Probleme Wissenslücken zu füllen und das Problem Codes zu kombinieren damit diese auch funktionieren. 

\subsubsection{Endpräsentation und Enddokumentation 14.01.22}
Bei der Endpräsentation haben wir uns vorgenommen früher vorzunehmen, aber wir konnten es aus zeitlichen Gründen nicht eine Woche früher vorbereiten.
Wir müssen nämlich auch den ChatBot ausbessern und einen Feinschliff durchführen. Die Präsentation kann nur vorbereitet werden, wenn der ChatBot fertig und vorführbar ist. 
Diese haben wir geplant am 11.01.22 fertig zu haben, damit wir noch bis zu der Präsentation üben können,
um die Präsentation so gut wie möglich vorzuführen
Die Enddokumentation wird schon von Anfang an verfasst und wird dann mit der Endpräsentation abgegeben.