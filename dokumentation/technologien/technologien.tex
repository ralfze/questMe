\section{Technologien}

\subsection{Technologie Versionen}

\subsubsection{Technologie Liste}

\begin{center}
\begin{tabular}{|c|c|}
\hline
\includegraphics[width=0.15\textwidth]{bilder/technologien/NodeJS.png} &
\multirow[c]{1}[1]{*}[30pt]{Node.js 14.18.0 LTS} \\
\hline
\includegraphics[width=0.15\textwidth]{bilder/technologien/Angular.png} &
\multirow[c]{1}[1]{*}[30pt]{Angular 12.2.9} \\
\hline
\includegraphics[width=0.15\textwidth]{bilder/technologien/Socket.io.png} &
\multirow[c]{1}[1]{*}[30pt]{Socket.io 4.2.0} \\
\hline
\includegraphics[width=0.15\textwidth]{bilder/technologien/mongoDB.png} &
\multirow[c]{1}[1]{*}[30pt]{MongoDB Community Edition 5.0.3} \\
\hline
\includegraphics[width=0.15\textwidth]{bilder/technologien/KeyCloak.png} &
\multirow[c]{1}[1]{*}[30pt]{KeyCloak 15.0.2} \\
\hline
\includegraphics[width=0.15\textwidth]{bilder/technologien/NLP.png} &
\multirow[c]{1}[1]{*}[30pt]{NLP.js 4.22.2 (optional)} \\
\hline
\end{tabular}
\end{center}

\subsubsection{Entscheidung für die Technologien}
Bei der Wahl der Technologien haben wir nach Möglichkeit LTS Versionen gesucht.
Damit wir eine Applikation erstellen können, die für lange Zeit Sicherheitsupdates
bekommt. Zusätzlich haben wir als Ziel, dass die Software als ingesamtes Paket sehr lange
stabil läuft und dadurch Ausfälle minimiert werden.

\subsubsection{NLP.js 4.22.2 (optional)}
Damit wir NLP.js nutzen können benötigen wir eine LTS Version von Node.js.
Diese Begrenzung haben wir zusätzlich in die Technologie Entscheidung einbezogen.
Damit die Integration von NLP.js in den Chatbot später immernoch möglich ist.

\subsection{Technologie Diagramme}

\begin{figure}[!hbt]
\centering
\includegraphics[width=1.0\textwidth]{bilder/technologien/Komponentendiagram v1.2.png}
\caption{Komponentendiagram v1.2}
\label{fig:Komponentendiagram_v1.2}
\end{figure}

\begin{figure}
\centering
\includegraphics[width=1.0\textwidth]{bilder/technologien/Komponentendiagram v1.1.png}
\caption{Komponentendiagram v1.1}
\label{fig:Komponentendiagram_v1.1}
\end{figure}

\begin{figure}
\centering
\includegraphics[width=1.0\textwidth]{bilder/technologien/Komponenten-Diagram-v1.png}
\caption{Komponentendiagram v1.0}
\label{fig:Komponentendiagram_v1.0}
\end{figure}
\FloatBarrier % prevent pictures from appearing under a different section