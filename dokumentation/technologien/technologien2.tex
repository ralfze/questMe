\subsection{Technologie Vergleich}
In diesem Bereich wollen wir unsere Design Entscheidung für eine Technologie durch einen Vergleich darstellen.
Dafür haben wir für die Themen ''Vergleich zwischen Angular und Vue.js'' und ''Vergleich zwischen MongoDB und PostgreSQL'' eine Tabelle erstellt.

\subsubsection{Vergleich zwischen Angular und Vue.js}

Als wir uns für ein Framework zur Entwicklung des Front-Ends entscheiden mussten, kamen Angular und Vue.js in unsere engere Auswahl.
Damit wir uns einen besseren Überblick über die Eigenschaften dieser Kandidaten verschaffen und ihre Vor- und Nachteile besser abwägen können, 
haben wir uns diese Tabelle erstellt.

\begin{center}
    \begin{table}[H]
        \begin{tabular}{|p{0.25\linewidth}|p{0.33\linewidth}|p{0.33\linewidth}|}
            \hline
            \textbf{}                                     & \textbf{Angular}                                     & \textbf{Vue.js}                       \\
            \hline
            \textbf{Framework,\newline Library, Platform} & Entwicklungsplattform\newline (Development platform) & Progressive Framework                 \\
            \hline
            \textbf{Gründer}                              & Google                                               & ehemaliger Google\newline Mitarbeiter \\
            \hline
            \textbf{Technology Typ}                       & MVC Framework                                        & MVVM Framework                        \\
            \hline
            \textbf{Programmier-\newline sprache}         & TypeScript                                           & JavaScript                            \\
            \hline
            \textbf{Performance}                          & niedrig                                              & hoch                                  \\
            \hline
            \textbf{Größe}                                & 500 kB                                               & 80 kB                                 \\
            \hline
            \textbf{Lernkurve}                            & Eine steile Lernkurve                                & Eine geringe Lernkurve                \\
            \hline
            \textbf{Dokumentation}                        & vorhanden                                            & vorhanden                             \\
            \hline
            \textbf{Datenbindung}                         & Bi-directional                                       & Bi-directional                        \\
            \hline
            \textbf{Rendering}                            & beim Client                                          & beim Server                           \\
            \hline
            \textbf{Code reuse}                           & möglich                                              & Ja, CSS und HTML                      \\
            \hline
            \textbf{Skalierbarkeit}                       & sehr hoch                                            & hoch                                  \\
            \hline
            \textbf{Testbarkeit}                          & mit einem Tool                                       & mehrere Tools benötigt                \\
            \hline
            \textbf{Vollständige Web\newline App}         & Kann als standalone Basis verwendet werden           & benötigt Third Party Tools            \\
            \hline
            \textbf{Lizenz}                               & MIT License                                          & MIT License                           \\
            \hline
        \end{tabular}
        \caption{Vergleich zwischen Angular und Vue.js}
        \label{Vergleich zwischen Angular und Vue.js}
    \end{table}
\end{center}

\noindent Unsere Entscheidung fiel schlussendlich auf Angular. 
Da es bereits länger auf dem Markt ist und auch einen höheren Marktanteil hat, ist es generell schon interessant es sich mal anzuschauen und damit zu arbeiten.
Hinzu kommt die sehr gute Testbarkeit, für die man lediglich ein einziges Tool benötigt. 
Ein großer Nachteil, der beim Blick auf die Tabelle sofort ins Auge sticht, ist aber die Performance.
Allerdings haben wir sie nur auf niedrig eingestuft, da am Anfang viel mitgeladen werden muss.
Bei größeren Projekten bietet Angular mehr Stabilität und Performance als seine Konkurrenz in diesem Vergleich.

\subsubsection{Vergleich zwischen MongoDB und PostgreSQL}

Ähnliche Gedanken wie für unser Front-End mussten wir uns auch für unser Back-End stellen. 
Genauer, welche Datenbank wollen wir verwenden? 
Da wir in der Datenbank den Korpus unseres Chatbots in Form von JSON-Dateien speichern wollen, 
haben wir uns zwei der populärsten Lösungen rausgesucht und diese in einer Tabelle gegenübergestellt.

\begin{center}
    \begin{table}[H]
        \begin{tabular}{|p{0.25\linewidth}|p{0.33\linewidth}|p{0.33\linewidth}|}
            \hline
            \textbf{}                                 & \textbf{MongoDB}        & \textbf{PostgreSQL}                         \\
            \hline
            \textbf{Primäres\newline Datenbankmodell} & Dokumentenorientiert    & Relationales DBMS                           \\
            \hline
            \textbf{Entwickler}                       & MongoDB. Inc            & PostgreSQL Global\newline Development Group \\
            \hline
            \textbf{Datenschema}                      & Schemafrei (NoSQL)      & Ja                                          \\
            \hline
            \textbf{Programmier-\newline sprachen}    & JavaScript + 28 weitere & JavaScript + 9 weitere                      \\
            \hline
            \textbf{Query Language}                   & MQL                     & SQL                                         \\
            \hline
            \textbf{Maximale\newline Dateigröße}      & 16 MB                   & 1 GB                                        \\
            \hline
            \textbf{Lernkurve}                        & Eine geringe Lernkurve  & Eine geringe Lernkurve                      \\
            \hline
            \textbf{Dokumentation}                    & vorhanden               & vorhanden                                   \\
            \hline
            \textbf{Skalierbarkeit}                   & sehr hoch               & hoch                                        \\
            \hline
            \textbf{Lizenz}                           & Open Source             & Open Source                                 \\
            \hline
        \end{tabular}
        \caption{Vergleich zwischen MongoDB und PostgreSQL}
        \label{tab:Vergleich zwischen MongoDB und PostgreSQL}
    \end{table}
\end{center}

\noindent Wir haben uns schlussendlich dafür entschieden, MongoDB zu nutzen. 
Da es bereits ein schemafreies und Dokumentenorientiertes Datenbankmodell ist, kommt uns das sehr gelegen.
Ein Nachteil, der sofort auffällt, ist die Maximale Dateigröße von gerade einmal 16 MB.
Da wir aber nicht davon ausgehen, dass unsere Dateien diese Größe erreichen werden, stellt das für uns kein Hindernis dar.