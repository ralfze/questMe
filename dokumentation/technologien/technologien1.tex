\section{Technologien}
In diesem Kapitel soll alles zum Thema Technologien zusammengefasst sein.
Zusätzlich sollen Schaubilder und Diagramme das Verständnis für die Technologien erleichtern.
Sowie Beschreibungen und Erklärungen zu der Wahl der einzelnen Technologien.

\subsection{Kriterien für die Technologien}
Bei der Wahl der Technologien haben wir nach Möglichkeit LTS Versionen gesucht.
Damit wir eine Applikation erstellen können, die für lange Zeit Sicherheitsupdates
bekommt. Zusätzlich haben wir als Ziel, dass die Software als insgesamtes Paket sehr lange
stabil läuft und dadurch Ausfälle minimiert werden.

\subsection{Technologie Vergleich}
In diesem Bereich wollen wir unsere Design Entscheidung für eine Technologie durch einen Vergleich darstellen.
Dafür haben wir für die Themen ''Vergleich zwischen Angular und Vue.js'' und ''Vergleich zwischen MongoDB und PostgreSQL'' eine Tabelle erstellt.

\newpage
\subsubsection{Vergleich zwischen Angular und Vue.js}

Als wir uns für ein Framework zur Entwicklung des Frontend entscheiden mussten, kamen Angular und Vue.js in unsere engere Auswahl.
Damit wir uns einen besseren Überblick über die Eigenschaften dieser Kandidaten verschaffen und ihre Vor- und Nachteile besser abwägen können, 
haben wir uns diese Tabelle erstellt.

\begin{center}
    \begin{table}[H]
        \begin{tabular}{|p{0.25\linewidth}|p{0.33\linewidth}|p{0.33\linewidth}|}
            \hline
            \textbf{}                                     & \textbf{Angular}                                                            & \textbf{Vue.js}                                               \\
            \hline
            \textbf{Framework,\newline Library, Platform} & \cellcolor{yellow!20}Entwicklungsplattform\newline (Development platform)   & \cellcolor{yellow!20}Progressive Framework                    \\
            \hline
            \textbf{Gründer}                              & \cellcolor{yellow!20}Google                                                 & \cellcolor{yellow!20}ehemaliger Google\newline Mitarbeiter    \\
            \hline
            \textbf{Technology Typ}                       & \cellcolor{green!20}MVC Framework                                           & \cellcolor{yellow!20}MVVM Framework                           \\
            \hline
            \textbf{Programmier-\newline sprache}         & \cellcolor{yellow!20}TypeScript                                             & \cellcolor{green!20}JavaScript                                \\
            \hline
            \textbf{Performance}                          & \cellcolor{yellow!20}niedrig                                                & \cellcolor{green!20}hoch                                      \\
            \hline
            \textbf{Größe}                                & \cellcolor{red!20}500 kB                                                    & \cellcolor{green!20}80 kB                                     \\
            \hline
            \textbf{Lernkurve}                            & \cellcolor{red!20}Eine steile Lernkurve                                     & \cellcolor{green!20}Eine geringe Lernkurve                    \\
            \hline
            \textbf{Dokumentation}                        & \cellcolor{green!20}vorhanden                                               & \cellcolor{green!20}vorhanden                                 \\
            \hline
            \textbf{Datenbindung}                         & \cellcolor{green!20}Bi-directional                                          & \cellcolor{green!20}Bi-directional                            \\
            \hline
            \textbf{Rendering}                            & \cellcolor{green!20}beim Client                                             & \cellcolor{yellow!20}beim Server                              \\
            \hline
            \textbf{Code reuse}                           & \cellcolor{green!20}möglich                                                 & \cellcolor{green!20}Ja, CSS und HTML                          \\
            \hline
            \textbf{Skalierbarkeit}                       & \cellcolor{green!20}sehr hoch                                               & \cellcolor{yellow!20}hoch                                     \\
            \hline
            \textbf{Testbarkeit}                          & \cellcolor{green!20}mit einem Tool                                          & \cellcolor{red!20}mehrere Tools benötigt                      \\
            \hline
            \textbf{Vollständige Web\newline App}         & \cellcolor{green!20}Kann als standalone Basis verwendet werden              & \cellcolor{red!20}benötigt Third Party Tools                  \\
            \hline
            \textbf{Lizenz}                               & \cellcolor{green!20}MIT License                                             & \cellcolor{green!20}MIT License                               \\
            \hline
        \end{tabular}
        \caption{Vergleich zwischen Angular und Vue.js}
        \label{Vergleich zwischen Angular und Vue.js}
    \end{table}
\end{center}

\noindent Unsere Entscheidung fiel schlussendlich auf Angular. 
Da es bereits länger auf dem Markt ist und auch einen höheren Marktanteil hat, ist es generell schon interessant es sich mal anzuschauen und damit zu arbeiten.
Hinzu kommt die sehr gute Testbarkeit, für die man lediglich ein einziges Tool benötigt. 
Die sehr hohe Skalierbarkeit und mögliche Wiederverwendbarkeit des Codes verspricht viele Möglichkeiten, auch für die Zukunft. 
Alle Teammitglieder haben bereits Erfahrung mit JavaScript als Programmiersprache, aber keine mit TypeScript. 
Der Umgang mit TypeScript muss also erst erlernt werden. 
Ein großer Nachteil, der beim Blick auf die Tabelle sofort ins Auge sticht, ist aber die Performance.
Allerdings haben wir sie nur auf niedrig eingestuft, da lediglich am Anfang viel mitgeladen werden muss.
Bei größeren Projekten bietet Angular mehr Stabilität und Performance als seine Konkurrenz in diesem Vergleich.

\subsubsection{Vergleich zwischen MongoDB und PostgreSQL}

Ähnliche Gedanken wie für unser Frontend mussten wir uns auch für unser Backend stellen. 
Genauer, welche Datenbank wollen wir verwenden? 
Da wir in der Datenbank den Korpus unseres Chatbots in Form von JSON-Dateien speichern wollen, 
haben wir uns zwei der populärsten Lösungen rausgesucht und diese in einer Tabelle gegenübergestellt.

\begin{center}
    \begin{table}[H]
        \begin{tabular}{|p{0.25\linewidth}|p{0.33\linewidth}|p{0.33\linewidth}|}
            \hline
            \textbf{}                                 & \textbf{MongoDB}                                & \textbf{PostgreSQL}                                               \\
            \hline
            \textbf{Primäres\newline Datenbankmodell} & \cellcolor{green!20}Dokumentenorientiert        & \cellcolor{yellow!20}Relationales DBMS                            \\
            \hline
            \textbf{Entwickler}                       & \cellcolor{yellow!20}MongoDB. Inc               & \cellcolor{yellow!20}PostgreSQL Global\newline Development Group  \\
            \hline
            \textbf{Datenschema}                      & \cellcolor{green!20}Schemafrei (NoSQL)          & \cellcolor{red!20}Ja                                              \\
            \hline
            \textbf{Programmier-\newline sprachen}    & \cellcolor{yellow!20}JavaScript + 28 weitere    & \cellcolor{yellow!20}JavaScript + 9 weitere                       \\
            \hline
            \textbf{Query Language}                   & \cellcolor{yellow!20}MQL                        & \cellcolor{green!20}SQL                                           \\
            \hline
            \textbf{Maximale\newline Dateigröße}      & \cellcolor{red!20}16 MB                         & \cellcolor{green!20}1 GB                                          \\
            \hline
            \textbf{Lernkurve}                        & \cellcolor{green!20}Eine geringe Lernkurve      & \cellcolor{green!20}Eine geringe Lernkurve                        \\
            \hline
            \textbf{Dokumentation}                    & \cellcolor{green!20}vorhanden                   & \cellcolor{green!20}vorhanden                                     \\
            \hline
            \textbf{Skalierbarkeit}                   & \cellcolor{green!20}sehr hoch                   & \cellcolor{yellow!20}hoch                                         \\
            \hline
            \textbf{Lizenz}                           & \cellcolor{green!20}Open Source                 & \cellcolor{green!20}Open Source                                   \\
            \hline
        \end{tabular}
        \caption{Vergleich zwischen MongoDB und PostgreSQL}
        \label{tab:Vergleich zwischen MongoDB und PostgreSQL}
    \end{table}
\end{center}

\noindent Wir haben uns schlussendlich dafür entschieden, MongoDB zu nutzen. 
Da es bereits ein schemafreies und Dokumentenorientiertes Datenbankmodell ist, kommt uns das sehr gelegen.
Beide Datenbankensysteme haben eine geringe Lernkurve und eine ähnlich gute Skalierbarkeit.
SQL ist zwar allen Teammitgliedern bereits bekannt, aber dank der umfangreichen Dokumentation von MongoDB sollte es leichtfallen, MQL zu erlernen.
Ein Nachteil, der sofort auffällt, ist die Maximale Dateigröße von gerade einmal 16 MB.
Da wir aber nicht davon ausgehen, dass unsere Dateien diese Größe erreichen werden, stellt das für uns kein Hindernis dar.